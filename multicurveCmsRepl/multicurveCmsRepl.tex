%% Based on a TeXnicCenter-Template by Gyorgy SZEIDL.
%%%%%%%%%%%%%%%%%%%%%%%%%%%%%%%%%%%%%%%%%%%%%%%%%%%%%%%%%%%%%

%------------------------------------------------------------
%
\documentclass{amsart}
%
%----------------------------------------------------------
% This is a sample document for the AMS LaTeX Article Class
% Class options
%        -- Point size:  8pt, 9pt, 10pt (default), 11pt, 12pt
%        -- Paper size:  letterpaper(default), a4paper
%        -- Orientation: portrait(default), landscape
%        -- Print size:  oneside, twoside(default)
%        -- Quality:     final(default), draft
%        -- Title page:  notitlepage, titlepage(default)
%        -- Start chapter on left:
%                        openright(default), openany
%        -- Columns:     onecolumn(default), twocolumn
%        -- Omit extra math features:
%                        nomath
%        -- AMSfonts:    noamsfonts
%        -- PSAMSFonts  (fewer AMSfonts sizes):
%                        psamsfonts
%        -- Equation numbering:
%                        leqno(default), reqno (equation numbers are on the right side)
%        -- Equation centering:
%                        centertags(default), tbtags
%        -- Displayed equations (centered is the default):
%                        fleqn (equations start at the same distance from the right side)
%        -- Electronic journal:
%                        e-only
%------------------------------------------------------------
% For instance the command
%          \documentclass[a4paper,12pt,reqno]{amsart}
% ensures that the paper size is a4, fonts are typeset at the size 12p
% and the equation numbers are on the right side
%
\usepackage{amsmath}%
\usepackage{amsfonts}%
\usepackage{amssymb}%
\usepackage{graphicx}
%------------------------------------------------------------
% Theorem like environments
%
\newtheorem{theorem}{Theorem}
\theoremstyle{plain}
\newtheorem{acknowledgement}{Acknowledgement}
\newtheorem{algorithm}{Algorithm}
\newtheorem{axiom}{Axiom}
\newtheorem{case}{Case}
\newtheorem{claim}{Claim}
\newtheorem{conclusion}{Conclusion}
\newtheorem{condition}{Condition}
\newtheorem{conjecture}{Conjecture}
\newtheorem{corollary}{Corollary}
\newtheorem{criterion}{Criterion}
\newtheorem{definition}{Definition}
\newtheorem{example}{Example}
\newtheorem{exercise}{Exercise}
\newtheorem{lemma}{Lemma}
\newtheorem{notation}{Notation}
\newtheorem{problem}{Problem}
\newtheorem{proposition}{Proposition}
\newtheorem{remark}{Remark}
\newtheorem{solution}{Solution}
\newtheorem{summary}{Summary}
\numberwithin{equation}{section}
%--------------------------------------------------------
\begin{document}
\title[Multicurve Replication of Cms Coupons]{Multicurve Replication of Cms Coupons}
\author{P. Caspers}
\email[P. Caspers]{pcaspers1973@googlemail.com}
\date{June 1, 2013}
\dedicatory{First Version June 1, 2013 - This Version June 1, 2013}
\begin{abstract}
We summarize formulas for the replication and pricing of Cms coupons in a multicurve setting.
\end{abstract}

\maketitle

\section{Notation}
We assume a two curve setting consisting of a discount curve $D$ and a forward curve $F$. All quantities computed on 
these curves are denoted with a respective subscript $D$ or $F$.

A swap rate fixed on $t_f$ may be written as

\begin{equation}
S(t_f) = \frac{\sum \tau_i L_F(t_f,t_i) P_D(t_f,s_i)}{\sum \tau^*_j P_D(t_f, s^*_j)}
\end{equation}

with $\tau$ denoting yearfractions, $t, s$ fixing and payment times, $L(u,v)$ the Libor rate fixed at $v$ as seen from u and $P$ a discount factor.

A swaption price $p$ for a swaption withh expiry on $t_f$ is given by the usual Black formula $B$

\begin{equation}
p = \sum \tau^*_j P_D(0,s^*_j) B( S(0), K, \sigma, t_f ) 
\end{equation}

\section{Curve Scenarios}
We use curve scenarios generated from a Hull White model to determine the replication basket. In a single curve setup we can write

\begin{equation}\label{curvescen}
P(t,t') = \frac{P(0,t')}{P(0,t)} e^{ -h G(t,t') }
\end{equation}

with $h = x + \phi$ derived from the normalized short rate $x(t) = r(t)-f(0,t)$ by a constant $\phi$ which only depends on the Hull White model parameters. Here

\begin{equation}
G(t,t') = \int_t^{t'} e^{-(u-t)\kappa} du = \frac{1-e^{-(t'-t)\kappa}}{\kappa}
\end{equation}

and for $\kappa=0$

\begin{equation}
G(t,t') = t' - t
\end{equation}

For our two curve setup we assume a static instantaneous forward spread between the curves, thus both curves have the same $x(t)$ and therefore equation \ref{curvescen} can be used for both curves $P_D$ and $P_F$. 

\section{Replication Basket}

\subsection{CMS Caplet}

We wish to replicate a CMS caplet with fixing time $t_f$, payment time $t_p$ and strike $K$. We assume a discretization $h_i, i=0,...,n$ with $h_0=K$.

The payoff of the caplet in scneario $i=1$ is $S(t_f,h_1)-K$ which discounts back to $t_f$ by $P_D(t_f,t_p,h_1)$. The npv of a physical settled call swaption with expiry on $t_f$ and strike $K$ is on the other hand $\Delta_1A(t_f,h_1)(S(t_f,h_1)-K)$ where $A$ denotes the annuity as written out explicitly above. Here $\Delta_1$ denotes the hedge weight. Equating the npv of the cms caplet payoff and the call swaption yields the first hedge weight

\begin{equation}
\Delta_1 = \frac{P_D(t_f,t_p,h_1)}{A(t_f,h_1)}
\end{equation}

We note here that it is possible to use a swaption price w.r.t. cash settlement where the annuity is replaced by

\begin{equation}
\sum_j \frac{\tau}{(1+\tau S(t_f))^ j}
\end{equation}

with a uniform $\tau$ corresponding to the frequency on the fixed leg.

For the second scenario we have to revalue the first hedge instrument as $\Delta_1A(t_f,h_2)(S(t_f,h_2)-K)$, subtract this from the npv of a cms caplet in scenario two which is $P_D(t_f,t_p,h_2)(S(t_f,h_2)-K)$ and solve for the weight $\Delta_2$ of the second hedge instrument which is a call swaption with strike $S(t_f,h_1)$ (therefore contributing nothing in the first scenario):

\begin{equation}
\Delta_2 = \frac{P_D(t_f,t_p,h_2)(S(t_f,h_2)-K) - \Delta_1A(t_f,h_2)(S(t_f,h_2)-K)}{A(t_f,h_2)(S(t_f,h_2)-S(t_f,h_1))}
\end{equation} 

We continue until the last scenario $i=n$.

\subsection{CMS Floorlet}

The case of a floorlet is handled analoguous to the caplet case.

\subsection{CMS Swaplet}

A swaplet is priced via parity.


\end{document}
