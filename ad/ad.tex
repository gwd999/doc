%% Based on a TeXnicCenter-Template by Gyorgy SZEIDL.
%%%%%%%%%%%%%%%%%%%%%%%%%%%%%%%%%%%%%%%%%%%%%%%%%%%%%%%%%%%%%

%------------------------------------------------------------
%
\documentclass{amsart}
%
%----------------------------------------------------------
% This is a sample document for the AMS LaTeX Article Class
% Class options
%        -- Point size:  8pt, 9pt, 10pt (default), 11pt, 12pt
%        -- Paper size:  letterpaper(default), a4paper
%        -- Orientation: portrait(default), landscape
%        -- Print size:  oneside, twoside(default)
%        -- Quality:     final(default), draft
%        -- Title page:  notitlepage, titlepage(default)
%        -- Start chapter on left:
%                        openright(default), openany
%        -- Columns:     onecolumn(default), twocolumn
%        -- Omit extra math features:
%                        nomath
%        -- AMSfonts:    noamsfonts
%        -- PSAMSFonts  (fewer AMSfonts sizes):
%                        psamsfonts
%        -- Equation numbering:
%                        leqno(default), reqno (equation numbers are on the right side)
%        -- Equation centering:
%                        centertags(default), tbtags
%        -- Displayed equations (centered is the default):
%                        fleqn (equations start at the same distance from the right side)
%        -- Electronic journal:
%                        e-only
%------------------------------------------------------------
% For instance the command
%          \documentclass[a4paper,12pt,reqno]{amsart}
% ensures that the paper size is a4, fonts are typeset at the size 12p
% and the equation numbers are on the right side
%
\usepackage{amsmath}%
\usepackage{amsfonts}%
\usepackage{amssymb}%
\usepackage{graphicx}
\usepackage[miktex]{gnuplottex}
\ShellEscapetrue
\usepackage{epstopdf}
\usepackage{longtable}
%------------------------------------------------------------
% Theorem like environments
%
\newtheorem{theorem}{Theorem}
\theoremstyle{plain}
\newtheorem{acknowledgement}{Acknowledgement}
\newtheorem{algorithm}{Algorithm}
\newtheorem{axiom}{Axiom}
\newtheorem{case}{Case}
\newtheorem{claim}{Claim}
\newtheorem{conclusion}{Conclusion}
\newtheorem{condition}{Condition}
\newtheorem{conjecture}{Conjecture}
\newtheorem{corollary}{Corollary}
\newtheorem{criterion}{Criterion}
\newtheorem{definition}{Definition}
\newtheorem{example}{Example}
\newtheorem{exercise}{Exercise}
\newtheorem{lemma}{Lemma}
\newtheorem{notation}{Notation}
\newtheorem{problem}{Problem}
\newtheorem{proposition}{Proposition}
\newtheorem{remark}{Remark}
\newtheorem{solution}{Solution}
\newtheorem{summary}{Summary}
\numberwithin{equation}{section}
%--------------------------------------------------------
\DeclareMathOperator{\sign}{sign}
%--------------------------------------------------------
\begin{document}
\title[AD]{Automatic Differentiation}
\author{P. Caspers}
\email[P. Caspers]{pcaspers1973@googlemail.com}
\date{May 31, 2013}
\dedicatory{First Version July 12, 2015 - This Version July 12, 2015\\INCOMPLETE DRAFT}
\begin{abstract}
We summarize the ideas underlying automatic differentiation.
\end{abstract}

\maketitle

\tableofcontents

\section{Computational Graph}

We consider the computation of a function $f: \mathbb{R}^n \rightarrow \mathbb{R}^m$ with independent variables $x_1, \dots , x_n$ and dependent variables $y_1, \dots , y_m$. The ultimate goal is to compute the Jacobian

\begin{equation}
J = \begin{pmatrix}
\frac{\partial y_1}{\partial x_1} &  & \dots &  & \frac{\partial y_1}{\partial x_n} \\
 & \ddots & & & \\
\vdots &  & \frac{\partial y_i}{\partial x_j} &  & \vdots \\
 &  &                                   & \ddots & \\
\frac{\partial y_m}{\partial x_1} & & \dots & & \frac{\partial y_m}{\partial x_n}
\end{pmatrix}
\end{equation}

We view the function $f$ as a composite of elementary operations

\begin{equation}
u_k = \Phi_k( \{u_\kappa\}_{\kappa \in \Lambda_k})
\end{equation}

for $k > n$ where we set $u_k = x_k$ for $k=1,\dots,n$ (i.e. we reserve these indices for the start values of the computation) and $u_k = y_k$ for $k=K-m+1, \dots, K$ (i.e. these are the final results of the computation). The notation should suggest that $u_k$ depends on prior results $u_\kappa$ with $\kappa$ in some index set $\Lambda_k$. Note that if $k\in\Lambda_l$ and $l\in\Lambda_m$ then $k\in\Lambda_m$ because we account for indirect dependencies also in our notation.

We can view the computation chain as a directed graph with vertices $u_k$ and edges $(k,l)$ if $k\in\Lambda_{l}$. There are no circles allowed in this graph and it consists of $K$ vertices. The graph is transitive closed in addition, cf. our previous comment on indirect dependencies above.

\section{Forward mode}

If we define the local partial derivative

\begin{equation}
c_{k,l} = \frac{\partial u_l}{\partial u_k}
\end{equation}

we can write the chain rule as

\begin{equation}\label{forward_u}
c_{k,m} = \frac{\partial u_m}{\partial u_k} = \sum_{l\in\Lambda_m} \frac{\partial u_m}{\partial u_l} \frac{\partial u_l}{\partial u_k} = \sum_{l\in\Lambda_m} c_{k,l}c_{l,m}
\end{equation}

This leads us to $c_{i,K-m+k} = \partial{y_k} / \partial{x_i}$ for $i=1,\dots,n$ and $k=K-m+1,\dots,K$, our desired final result, which we can compose from elementary local partial derivatives using \ref{forward_u}.

\section{Backward mode}

With so called adjoint variables $\overline{c_{\cdot,\cdot}}$ we define

\begin{equation}\label{adjoint_u}
\overline{c_{l,m}} = \sum_{m\in\Lambda^{-1}_l} \frac{\partial u_m}{\partial u_l} \overline{c_{m,m}} 
\end{equation}

where $\Lambda^{-1}_l$ is the set of indices $m$ with $(l,m)\in\Lambda_m$, i.e. the $m$ for which $u_m$ depends on $u_l$. Note that the ``direction'' of equation \ref{adjoint_u} is opposite to \ref{forward_u} in the sense that $\overline{u}_l$ is computed from values $\overline{u}_m$ instead of the other way around, which is the ``natural'' direction of the chain rule.

We apply the chain rule to \ref{adjoint_u} which yields

\begin{equation}
\overline{c_{k,m}} = \sum_{m\in\Lambda^{-1}_l} \frac{\partial u_m}{\partial u_l} \frac{\partial u_l}{\partial u_k} \overline{c_{m,m}} = \sum_{m\in{\Lambda_l^{-1}}} c_{k,l} \overline{c_{l,m}}
\end{equation}

Again this yields $\overline{c_{i,k}} = \partial{y_k}/\partial{x_i}$ for $i=1,\dots,n$ and $k=K-m+1,\dots,K$ as above, if we set $\overline{c_{k,k}} = 1$ for the desired $k$ and zero otherwise, now enabling us to compute the final result working backwards through the elememtary local partial derivatives.




\begin{thebibliography}{2}

\bibitem{ql}QuantLib A free/open-source library for quantitative finance, http://www.quantlib.org

\end{thebibliography}

\end{document}

