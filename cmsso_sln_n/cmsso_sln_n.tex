%% Based on a TeXnicCenter-Template by Gyorgy SZEIDL.
%%%%%%%%%%%%%%%%%%%%%%%%%%%%%%%%%%%%%%%%%%%%%%%%%%%%%%%%%%%%%

%------------------------------------------------------------
%
\documentclass{amsart}
%
%----------------------------------------------------------
% This is a sample document for the AMS LaTeX Article Class
% Class options
%        -- Point size:  8pt, 9pt, 10pt (default), 11pt, 12pt
%        -- Paper size:  letterpaper(default), a4paper
%        -- Orientation: portrait(default), landscape
%        -- Print size:  oneside, twoside(default)
%        -- Quality:     final(default), draft
%        -- Title page:  notitlepage, titlepage(default)
%        -- Start chapter on left:
%                        openright(default), openany
%        -- Columns:     onecolumn(default), twocolumn
%        -- Omit extra math features:
%                        nomath
%        -- AMSfonts:    noamsfonts
%        -- PSAMSFonts  (fewer AMSfonts sizes):
%                        psamsfonts
%        -- Equation numbering:
%                        leqno(default), reqno (equation numbers are on the right side)
%        -- Equation centering:
%                        centertags(default), tbtags
%        -- Displayed equations (centered is the default):
%                        fleqn (equations start at the same distance from the right side)
%        -- Electronic journal:
%                        e-only
%------------------------------------------------------------
% For instance the command
%          \documentclass[a4paper,12pt,reqno]{amsart}
% ensures that the paper size is a4, fonts are typeset at the size 12p
% and the equation numbers are on the right side
%
\usepackage{amsmath}%
\usepackage{amsfonts}%
\usepackage{amssymb}%
\usepackage{graphicx}
\usepackage[miktex]{gnuplottex}
\ShellEscapetrue
\usepackage{epstopdf}
\usepackage{longtable}
%------------------------------------------------------------
% Theorem like environments
%
\newtheorem{theorem}{Theorem}
\theoremstyle{plain}
\newtheorem{acknowledgement}{Acknowledgement}
\newtheorem{algorithm}{Algorithm}
\newtheorem{axiom}{Axiom}
\newtheorem{case}{Case}
\newtheorem{claim}{Claim}
\newtheorem{conclusion}{Conclusion}
\newtheorem{condition}{Condition}
\newtheorem{conjecture}{Conjecture}
\newtheorem{corollary}{Corollary}
\newtheorem{criterion}{Criterion}
\newtheorem{definition}{Definition}
\newtheorem{example}{Example}
\newtheorem{exercise}{Exercise}
\newtheorem{lemma}{Lemma}
\newtheorem{notation}{Notation}
\newtheorem{problem}{Problem}
\newtheorem{proposition}{Proposition}
\newtheorem{remark}{Remark}
\newtheorem{solution}{Solution}
\newtheorem{summary}{Summary}
\numberwithin{equation}{section}
%--------------------------------------------------------
\DeclareMathOperator{\sign}{sign}
%--------------------------------------------------------
\begin{document}
\title[Cms spread option pricing with normal and shifted lognormal dynamics]{Cms spread option pricing with normal and shifted lognormal dynamics}
\author{P. Caspers}
\email[P. Caspers]{pcaspers1973@googlemail.com}
\date{October 29, 2015}
\dedicatory{First Version October 29, 2015 - This Version October 29, 2015}
\begin{abstract}
We extend the cms spread option formula in \cite{brigo}, 13.16.2 to the case of shifted lognormal or normal swap rate dynamics.
\end{abstract}

\maketitle

\tableofcontents

\section{Original lognormal model}

The original model in \cite{brigo}, 13.16.2 for cms spread option pricing is as follows. If $S_i, i=1,2$ are the underlying swap rates, $t$ is the fixing time, $T$ the payment time of the coupon with payoff ($\phi\in\{-1,+1\}$)

\begin{equation}
\Pi(t) = \max( \phi (a S_1(t) + b S_2(t) - K), 0 )
\end{equation}

then under the $T$-forward measure we assume

\begin{eqnarray}\label{origmodel}
dS_1 &=& \mu_1 S_1 dt + \sigma_1 S_1 dZ_1 \\
dS_2 &=& \mu_2 S_2 dt + \sigma_2 S_2 dZ_2 \\
dZ_1 dZ_2 &=& \rho dt
\end{eqnarray}

\cite{brigo} gives a formula for the price $P(0,T) E^T ( \Pi(t) )$ involving a one dimensional integral as formula (13.34). The drifts $\mu_i$ are implied from exogeneously given convexity adjustments for $S_i$, e.g. computed in a replication model on the respective underlying smiles).

Note that the formulation is \cite{brigo} is a little more restrictive, because there it is assumed $t=T$. However, provided that the swap rate adjustments implying $\mu_i, i=1,2$ are computed w.r.T. the same fixing time $t$ and payment time $T>t$, one can just replace the pricing expectation $E^t$ by $E^T$ without changing anything (inside the expectation a factor $P(t,T) / P(t,T) = 1$ occurs, which doesn't change anything).

\section{Shifted lognormal extension}

We look at the following straightforward extension of the model, introducing shifts for the underlying rates

\begin{eqnarray}
dS_1 &=& \mu_1 (S_1 + d_1) dt + \sigma_1 (S_1 + d_1) dZ_1 \\
dS_2 &=& \mu_2 (S_2 + d_2) dt + \sigma_2 (S_2 + d_2) dZ_2 \\
dZ_1 dZ_2 &=& \rho dt
\end{eqnarray}

with $d_1, d_2 \geq 0$, possibly different. Writing

\begin{eqnarray}
X_1 &=& (S_1 + d_1) \\
X_2 &=& (S_2 + d_2) \\
L &=& K + a d_1 + b d_2 
\end{eqnarray}

we see that we can apply the original solution for \ref{origmodel} with underlyings $X_1, X_2$ and the payoff written as $(aX_1+bX_2-L)^+$. Note that the computation of the drifts $\mu_i$ change to 

\begin{equation}
\mu_i = \frac{\log( (S_i(0)+d_i+c_i) / (S_i(0)+d_i) )}{t} = \frac{\log( (X_i(0)+c_i) / (X_i(0)) )}{t} 
\end{equation}

for $i=1,2$ accordingly with $c_i$ denoting the convexity adjustment applicable to rate $S_i$.

\section{Normal extension}

The normal flavour of the original model reads

\begin{eqnarray}
dS_1 &=& \mu_1 dt + \sigma_1 dZ_1 \\
dS_2 &=& \mu_2 dt + \sigma_2 dZ_2 \\
dZ_1 dZ_2 &=& \rho dt
\end{eqnarray}

with drifts now given by

\begin{equation}
\mu_i = c_i / t
\end{equation}

with $c_i$ again denoting the exogeneously given convexity adjustment for rate $S_i, i=1,2$.

The option price $\nu$ is given by

\begin{equation}
\nu = P(0,T) E^T \left( (\phi(aS_1(t) + bS_2(t) - K))^+ \right)
\end{equation}

The expectation can, more explicity, written as

\begin{equation}\label{normalintegral}
\int_{-\infty}^{\infty} \int_{-\infty}^{\infty} (\phi(as_1+bs_2-K))^+ p(s_1,s_2) ds_1 ds_2
\end{equation}

with $p$ denoting a bivariate normal distribution with expectation $\mu$ and covariance matirx $\Sigma$

\begin{equation}
\mu = \begin{pmatrix} \mu_1t \\ \mu_2t \end{pmatrix}, 
\Sigma = \begin{pmatrix} \sigma_1^2t & \rho\sigma_1\sigma_2t \\ \rho\sigma_1\sigma_2t & \sigma_2^2t \end{pmatrix}
\end{equation}

It is well known that the distribution of $S_1(t)$ conditional on $S_2(t) = s_2$ is given by

\begin{equation}
S_1(t) | \{ S_2(t) = s_2 \} \sim \mathcal{N}\left( \mu_1t + \frac{\rho\sigma_1}{\sigma_2}(s_2-\mu_2t) , \sigma_1^2t (1-\rho^2)\right)
\end{equation}

We denote the density of this distribution by $p_{s_2}$ We continue to compute the inner integral in \ref{normalintegral} for a fixed $S_2(t) = s_2$, i.e.

\begin{equation}\label{innerintegral}
\int_{-\infty}^{\infty} (\phi(as_1+bs_2-K))^+ p_{s_2}(s_1) ds_1
\end{equation}

We have the following

\begin{lemma}
\label{linearnormalintegral}
For $\alpha, \beta \in \mathbb{R}$ we have
\begin{equation}
\frac{1}{\sqrt{2 \pi}} \int_{\mathbb{R}} (\alpha x+\beta)^+ e^{-x^2/2} dx = \phi \frac{\alpha}{\sqrt{2\pi}}e^{-\beta^2/(2\alpha^2)} + \beta \phi (1-N(-\beta/\alpha))
\end{equation}
where $\phi = \sign(\alpha) \in \{+1,-1\}$ and $N$ denotes the cumulative normal distribution function. For $\alpha=0$ the right hand side simplifies to $\beta^+$.
\end{lemma}
\begin{proof}
Let $\alpha>0$. The proof for $\alpha<0$ is similar and the case $\alpha=0$ is obvious from the following steps as well. Obviously $\alpha x+\beta > 0$ iff $x > -\beta/\alpha$, so the integral can be written as
\begin{equation}
\frac{1}{\sqrt{2 \pi}} \int_{-\beta/\alpha}^{\infty} (\alpha x+\beta) e^{-x^2/2} dx 
\end{equation}
Furthermore,
\begin{equation}
\int_{-\beta/\alpha}^\infty x e^{-x^2/2} dx = \left[ -e^{-x^2/2} \right]_{-\beta/\alpha}^\infty = e^{-\beta^2/(2\alpha^2)}
\end{equation}
and
\begin{equation}
\int_{-\beta/\alpha}^\infty e^{-x^2/2} dx = 1 - N(-\beta/\alpha)
\end{equation}
which proves the identity.
\end{proof}

THe integral \ref{innerintegral} is

\begin{equation}
\frac{1}{\sqrt{2\pi}} v \int_{-\infty}^\infty (\phi(as_1+bs_2-K))^+ e^{-\frac{(s1-\mu)^2}{2v^2}} ds_1
\end{equation}

with $\mu=\mu_1t+\rho\sigma_1/\sigma_2(s_2-\mu_2t)$ and $v^2=\sigma_1^2t(1-\rho^2)$.

Substituting $x=(s_1-\mu)/v$ this becomes

\begin{equation}
\frac{1}{\sqrt{2\pi}}\int_{-\infty}^{\infty} (\phi a(\mu+vx)+bs_2-K)^+ e^{-x^2/2} dx
\end{equation}

Setting $\alpha=\phi a v$ and $\beta=\phi a \mu + b s_2 - K$ and $\psi = \sign(\alpha)$ we can write the npv (using lemma \ref{linearnormalintegral}) $\nu$ as 

\begin{equation}
P(0,T) \frac{1}{\sigma_2\sqrt{2\pi t}} \int_{-\infty}^\infty \psi\left[\frac{\alpha}{\sqrt{2\pi}}e^{-\beta^2/(2\alpha^2)}+\beta(1-N(-\beta/\alpha))\right]e^{-\frac{(s_2-\mu_2t)^2}{2\sigma_2^2 t}} ds_2
\end{equation}

with

\begin{eqnarray}
\alpha &=& \phi a \sigma_1 \sqrt{t(1-\rho^2)} \\
\beta &=& \phi a (\mu_1 t + \rho\sigma_1/\sigma_2(s_2-\mu_2t)) + bs_2 - K \\
\psi &=& \sign(\alpha)
\end{eqnarray}

A final substitution $s = (s_2 - \mu_2t) / (\sigma_2\sqrt{t})$ yields

\begin{equation}
\nu = P(0,T) \frac{1}{\sigma_2\sqrt{2\pi t}} \int_{-\infty}^\infty \psi\left[\frac{\alpha}{\sqrt{2\pi}}e^{-\beta^2/(2\alpha^2)}+\beta(1-N(-\beta/\alpha))\right]e^{-s^2/2} ds
\end{equation}

with

\begin{eqnarray}
\alpha &=& \phi a \sigma_1 \sqrt{t(1-\rho^2)} \\
\beta &=& \phi a (\mu_1 t + \rho\sigma_1/\sigma_2^2s\sqrt{t}) + b(s\sigma_2\sqrt{t}+\mu_2t) - K \\
\psi &=& \sign(\alpha)
\end{eqnarray}


\begin{thebibliography}{1}
\bibitem{brigo}Brigo, Mercurio: Interst Rate Models - Theory and Practice, 2nd Edition, Springer, 2006
\end{thebibliography}

\end{document}

