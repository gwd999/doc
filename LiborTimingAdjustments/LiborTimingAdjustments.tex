%% Based on a TeXnicCenter-Template by Gyorgy SZEIDL.
%%%%%%%%%%%%%%%%%%%%%%%%%%%%%%%%%%%%%%%%%%%%%%%%%%%%%%%%%%%%%

%------------------------------------------------------------
%
\documentclass{amsart}
%
%----------------------------------------------------------
% This is a sample document for the AMS LaTeX Article Class
% Class options
%        -- Point size:  8pt, 9pt, 10pt (default), 11pt, 12pt
%        -- Paper size:  letterpaper(default), a4paper
%        -- Orientation: portrait(default), landscape
%        -- Print size:  oneside, twoside(default)
%        -- Quality:     final(default), draft
%        -- Title page:  notitlepage, titlepage(default)
%        -- Start chapter on left:
%                        openright(default), openany
%        -- Columns:     onecolumn(default), twocolumn
%        -- Omit extra math features:
%                        nomath
%        -- AMSfonts:    noamsfonts
%        -- PSAMSFonts  (fewer AMSfonts sizes):
%                        psamsfonts
%        -- Equation numbering:
%                        leqno(default), reqno (equation numbers are on the right side)
%        -- Equation centering:
%                        centertags(default), tbtags
%        -- Displayed equations (centered is the default):
%                        fleqn (equations start at the same distance from the right side)
%        -- Electronic journal:
%                        e-only
%------------------------------------------------------------
% For instance the command
%          \documentclass[a4paper,12pt,reqno]{amsart}
% ensures that the paper size is a4, fonts are typeset at the size 12p
% and the equation numbers are on the right side
%
\usepackage{amsmath}%
\usepackage{amsfonts}%
\usepackage{amssymb}%
\usepackage{graphicx}
%------------------------------------------------------------
% Theorem like environments
%
\newtheorem{theorem}{Theorem}
\theoremstyle{plain}
\newtheorem{acknowledgement}{Acknowledgement}
\newtheorem{algorithm}{Algorithm}
\newtheorem{axiom}{Axiom}
\newtheorem{case}{Case}
\newtheorem{claim}{Claim}
\newtheorem{conclusion}{Conclusion}
\newtheorem{condition}{Condition}
\newtheorem{conjecture}{Conjecture}
\newtheorem{corollary}{Corollary}
\newtheorem{criterion}{Criterion}
\newtheorem{definition}{Definition}
\newtheorem{example}{Example}
\newtheorem{exercise}{Exercise}
\newtheorem{lemma}{Lemma}
\newtheorem{notation}{Notation}
\newtheorem{problem}{Problem}
\newtheorem{proposition}{Proposition}
\newtheorem{remark}{Remark}
\newtheorem{solution}{Solution}
\newtheorem{summary}{Summary}
\numberwithin{equation}{section}
%--------------------------------------------------------

\begin{document}
\title[Libor Timing Adjustments]{Libor Timing Adjustments}
\author{Peter Caspers}
\email[Peter Caspers]{pcaspers1973@gmail.com}
\date{December 10, 2012}
\dedicatory{First Version November 1, 2012 - This Version August 4, 2015}
\begin{abstract}
We derive a closed form expression for the convexity adjustment to be applied to a Libor coupon with non natural
payment time. The model is a two dimensional lognormal model for the Libor rate and a forward rate
naturally associated to this rate and the payment time of the coupon. In particular we recover the in arrears
fixing adjustment as a special case.
\end{abstract}

\maketitle

\section{Introduction}

Consider a Libor rate with fixing time $t_f$, index start date $t_s$ and index end date $t_e$. For Euribor indices
$t_f$ is e.g. deduced from $t_s$ by subtracting two target business days while $t_e$ is obatained by adding
the Libor index tenor to $t_s$ and adjusting to a business day using the rolling convention modified following (following in case of weekly tenors).
Note that it is quite usual e.g. in a vanilla interest rate swap to have $t_e > t_p$ for some floating coupons in the schedule where $t_p$ denotes the payment
time of the Libor coupon, though the swap has natural payment lags. The convexity adjustment to be applied in this case
is of course very small, because $t_e - t_p$ is at most a few days. Therefore the adjustment usually is ignored. However
if the payment is made in arrears, i.e. $t_p = t_s$ or is delayed significantly (more than a few days), i.e. $t_p \gg t_e$, or if an index is
averaged over some fixings, i.e. $t_s < t_p < t_e$ for (some of the) contributions to the average, then the adjustment
becomes usually non negligible and should therefore be included in the coupon pricing. We derive an easy closed form expression for this timing adjustment
applicable to all cases mentioned above based on a simple two dimensional lognormal model for the rate to be adjusted and a forward rate associated
to this rate and the payment time $t_p$.

\section{A general adjustment formula}

With notations as above and $\tau_{\textnormal{accr}}$ denoting the year fraction of the accrual period (usually starting on $t_s$ and
ending on $t_p$, which is however completely unimportant for the following derivations) we can price the coupon using
the general pricing formula

\begin{equation}\label{generaladjustment}
C = P(0,t_p) E^{t_p} \left( \frac{F(t_f)\tau_{\textnormal{accr}} P(t_f,t_p)}{P(t_f,t_p)} \right) = P(0,t_p) E^{t_p} ( F(t_f) \tau_{\textnormal{accr}} )
\end{equation}

with $F(t)$ denoting the forward Libor rate, $P(t,T)$ denoting the price at time $t$ of a zero bond with maturity $T$
and the expectation being taken in the $t_p$-forward measure $\mathbb{Q}^{t_p}$. We note here that our derivation will be done in a single
curve setup for the moment, i.e. with identical forwading and discounting curves.

We see that the adjusted rate is simply given by the $t_p$-forward expectation of the rate evaluated on the fixing time. It is
then clear, that this is equal to the time zero forward rate F(0) if (and only if) $t_p = t_e$, because $F(t)$ is an $\mathbb{Q}^{t_e}$-martingale.
The latter is true since

\begin{equation}\label{forward1}
F(t) = \frac{1}{\tau} \left( \frac{P(t,t_s)}{P(t,t_e)} -1 \right)   
\end{equation}

with $F$ simply compounding and $\tau$ now (and in the following) refering to the year fraction from $t_s$ to $t_e$, i.e. the index estimation period (and not the accrual period as the $\tau$ above).

The convexity adjustment is obviously model dependent. We will now derive a closed
form expression for the convexity adjustment in a particularly basic model.

\section{Lognormal adjustment}

In addition to \ref{forward1} we introduce a second forward rate $F^*$ with estimation start date $t_s^*$ and estimation end
date $t_e^* = t_p$ equal to the payment time of our coupon in question. We choose $t_s^* = t_e$ if $t_p > t_e$ (i.e. if we have
a \textit{delayed} payment) and $t_s^* = t_s$ otherwise (i.e. if we have an \textit{early} payment).

We let these two forward rates evolve according to 

\begin{eqnarray}\label{model}
dF &=& \sigma F dW \\
dF^* &=& \sigma^* F^* dW^* \\
dW dW^* &=& \rho dt
\end{eqnarray}

where $dW$ and $dW^*$ are brownian motions in the respective forward measures associated to $F$ and $F^*$ denoted by $\mathbb{Q}$ and $\mathbb{Q}^*$
in the following. To compute the adjusted forward we follow \ref{generaladjustment} and therefore aim to compute

\begin{equation}
E^*( F(t_f) )
\end{equation}

with the expectation taken in the measure $\mathbb{Q}^*$. The derivation is now following the (very nice) elaboration on the Girsanov Theorem in \cite{Papa}.
Since

\begin{equation}
P(0,t_e) E\left( \frac{C(t)}{P(t,t_e)} \right) = P(0,t_p) E^*\left( \frac{C(t)}{P(t,t_p)} \right)
\end{equation}

for an arbitrary payoff $C$ we get the Radon Nikodym derivative

\begin{equation}
Z(t) = \left. \frac{d\mathbb{Q}^*}{d\mathbb{Q}} \right|_{\mathcal{F}_t} = \frac{P(0,t_e)P(t,t_p)}{ P(0,t_p)P(t,t_e)}
\end{equation}

which in terms of forward rates reads

\begin{equation}
Z(t) = \frac{P(0,t_e)}{P(0,t_p)} \frac{(1+\tau_F F(t)\textbf{1}_{t_p < t_e})}{1+\tau_{F^*} F^*(t)}
\end{equation}

with $\textbf{1}_A$ denoting the indicator function. We stress again that $\tau_X$ is the year fraction belonging to the index estimation period
of the respective forward $X$.

%It is useful at this stage to switch to log forwards through which formulas \ref{model} and following become (thanks to Ito)
%
%\begin{eqnarray}
%d \ln{f} = \sigma dW - \frac{1}{2} \sigma^2 dt \\
%d \ln{f^*} = \sigma^* dW^* - \frac{1}{2} \sigma^{*2} dt \\
%dW dW^* = \rho dt
%\end{eqnarray}

We get $d\ln{Z}$ from that as

\begin{equation}
d\ln{Z(t)} = d\ln{(1+\tau_F F(t)\textbf{1}_{t_p < t_e})} - d\ln{(1+\tau_{F^*} F^*(t))}
\end{equation}

so the instantaneous quadratic covariation between $d\ln{Z}$ and $d\ln{F}$ is given by

\begin{equation}
\sigma dW (d\ln{(1+\tau_F F(t)\textbf{1}_{t_p < t_e})} - d\ln{(1+\tau_{F^*} F^*(t))})
\end{equation}

Observing that

\begin{equation}
d\ln(1+\tau_F F(t)) = \frac{d(1+\tau_F F(t))}{1+\tau_F F(t)} + \ldots dt
\end{equation}

which holds similarly for $F^*$ the covariation expression becomes

\begin{equation}
\sigma dW \left( \frac{d(1+\tau_F F(t))}{1+\tau_F F(t)}\textbf{1}_{t_p < t_e} - \frac{d(1+\tau_F^* F^*(t))}{1+\tau_F^* F^*(t)} \right) 
\end{equation}

which evaluates to

\begin{equation}
\left(\frac{\tau_F \sigma^2 F(t)}{1+\tau_F F(t)} \textbf{1}_{t_p<t_e} - \frac{\tau_F^* \rho\sigma\sigma^* F^*(t)}{1+\tau_F^* F^*(t)}\right) dt
\end{equation}

and finally yielding the dynamics of $F$ under the measure $\mathbb{Q}^*$

\begin{equation}\label{tpdynamic}
dF(t) = F(t) \left(\frac{\tau_F \sigma^2 F(t)}{1+\tau_F F(t)} \textbf{1}_{t_p<t_e} - \frac{\tau_{F^*} \rho\sigma\sigma^* F^*(t)}{1+\tau_{F^*} F^*(t)}\right) dt + F(t) \sigma dW^\prime
\end{equation}

with a $\mathbb{Q^*}$ - Brownian motion $W^\prime$.

To get a closed formula for the adjusted rate we have to make the drift term in formula \ref{tpdynamic} deterministic. A standard approach to do this
is to freeze the forwards at time zero. Doing so we get the convexity adjustment

\begin{equation}
E^*(F(t_f)) - F(0) \approx t_f \left(\frac{\tau_F \sigma^2 F^2(0)}{1+\tau_F F(0)} \textbf{1}_{t_p<t_e} - \frac{\tau_{F^*} \rho\sigma\sigma^*F(0) F^*(0)}{1+\tau_{F^*} F^*(0)}\right)
\end{equation}

The Libor in arrears adjustment from \cite{Hfb} is recovered as a special case with $t_p=t_s$, the second term inside the brackets then vanishing (because $\tau_{F^*}=0$).

It should be noted that there is a restriction for the model parameter $\rho$ in the case $t_p < t_e$ which we not further investigate here for now. In particular we note that $\rho \rightarrow 1$ necessarily if $t_p \rightarrow t_e$ from the left. Since if $t_p < t_e$ implies that the forward rates $F$ and
$F^*$ are overlapping with same start date and a low distance between their end dates, most likely one will choose $\rho$ near or equal $1$ anyway.

Furthermore obviously the convexity adjustment converges to zero both for $t_p \rightarrow t_e$ from the left and from the right. As a result the convexity
adjustment is a continuous function in $t_p$ (with all other variables fixed) with a zero at $t_p = t_e$. This is both expected and a good property.

Concerning the model parameter $\sigma$ the obvious choice is the ATM caplet volatility for the index associated with $F$. For $\sigma^*$ some average
over the same index caplet volatilities spanning the period of $F^*$ can be choosen, possibly accounting for correlations between the respective forward rates. Similarly the model parameter $\rho$ can be taken from some market model with swaption and / or cms spread option calibrated correlation structure. Again, we do not go into details here for now.

\section{Open Points / To Do}

What is the restriction on $\rho$ if $t_p < t_e$  to ensure an arbitrage free model dynamics in \ref{model} ? What is a good choice for $\sigma^*$ and $\rho$  ? Can a replication adjustment be calculated in the general setting of this paper taking into account the volatility smile ? Generalize the derivation to the case of multicurve curve stripping, i.e. with two distinct curves for discounting and forwarding.




\begin{thebibliography}{2}

\bibitem {Papa}Papaioannou Denis, \textit{Applied Multidimensional Girsanov Theorem},
Electronic copy available at: http://ssrn.com/abstract=1805984

\bibitem {Hfb}Heidorn, Schmidt, \textit{LIBOR in Arrears}, 1998, Frankfurt School of Finance

\end{thebibliography}

\end{document}