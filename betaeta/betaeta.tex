%% Based on a TeXnicCenter-Template by Gyorgy SZEIDL.
%%%%%%%%%%%%%%%%%%%%%%%%%%%%%%%%%%%%%%%%%%%%%%%%%%%%%%%%%%%%%

%------------------------------------------------------------
%
\documentclass{amsart}
%
%----------------------------------------------------------
% This is a sample document for the AMS LaTeX Article Class
% Class options
%        -- Point size:  8pt, 9pt, 10pt (default), 11pt, 12pt
%        -- Paper size:  letterpaper(default), a4paper
%        -- Orientation: portrait(default), landscape
%        -- Print size:  oneside, twoside(default)
%        -- Quality:     final(default), draft
%        -- Title page:  notitlepage, titlepage(default)
%        -- Start chapter on left:
%                        openright(default), openany
%        -- Columns:     onecolumn(default), twocolumn
%        -- Omit extra math features:
%                        nomath
%        -- AMSfonts:    noamsfonts
%        -- PSAMSFonts  (fewer AMSfonts sizes):
%                        psamsfonts
%        -- Equation numbering:
%                        leqno(default), reqno (equation numbers are on the right side)
%        -- Equation centering:
%                        centertags(default), tbtags
%        -- Displayed equations (centered is the default):
%                        fleqn (equations start at the same distance from the right side)
%        -- Electronic journal:
%                        e-only
%------------------------------------------------------------
% For instance the command
%          \documentclass[a4paper,12pt,reqno]{amsart}
% ensures that the paper size is a4, fonts are typeset at the size 12p
% and the equation numbers are on the right side
%
\usepackage{amsmath}%
\usepackage{amsfonts}%
\usepackage{amssymb}%
\usepackage{graphicx}
\usepackage[miktex]{gnuplottex}
\ShellEscapetrue
\usepackage{epstopdf}
\usepackage{longtable}
%------------------------------------------------------------
% Theorem like environments
%
\newtheorem{theorem}{Theorem}
\theoremstyle{plain}
\newtheorem{acknowledgement}{Acknowledgement}
\newtheorem{algorithm}{Algorithm}
\newtheorem{axiom}{Axiom}
\newtheorem{case}{Case}
\newtheorem{claim}{Claim}
\newtheorem{conclusion}{Conclusion}
\newtheorem{condition}{Condition}
\newtheorem{conjecture}{Conjecture}
\newtheorem{corollary}{Corollary}
\newtheorem{criterion}{Criterion}
\newtheorem{definition}{Definition}
\newtheorem{example}{Example}
\newtheorem{exercise}{Exercise}
\newtheorem{lemma}{Lemma}
\newtheorem{notation}{Notation}
\newtheorem{problem}{Problem}
\newtheorem{proposition}{Proposition}
\newtheorem{remark}{Remark}
\newtheorem{solution}{Solution}
\newtheorem{summary}{Summary}
\numberwithin{equation}{section}
%--------------------------------------------------------
\DeclareMathOperator{\sign}{sign}
%--------------------------------------------------------
\begin{document}
\title[$\beta-\eta$ Model Implementation]{$\beta-\eta$ Model Implementation}
\author{P. Caspers}
\author{R. Lichters}
\email[P. Caspers]{pcaspers1973@googlemail.com}
\email[R. Lichters]{pcaspers1973@googlemail.com}
\date{May 31, 2013}
\dedicatory{First Version July 10, 2015 - This Version July 10, 2015\\INCOMPLETE DRAFT}
\begin{abstract}
We describe the implementation of the $\beta-\eta$ model (\cite{betaeta}, \cite{piterbarg}, 11.3.2.6) in QuantLib \cite{ql}.
\end{abstract}

\maketitle

\section{Model}

The driving process is given by

\begin{equation}\label{dynamics}
dx(t) = \alpha(t) ( 1 + \beta x(t) )^\eta dW(t) 
\end{equation}

with $x(0)=0$, $\alpha(\cdot) > 0$, $\beta > 0$ and $0 \leq \eta \leq 1$. The dynamics is expressed in the measure $\mathbb{Q}^N$ associated to the numeraire

\begin{equation}
N(t,x(t)) = \frac{1}{P(0,t)}e^{\lambda(t)x(t)+M(0,0;T)}
\end{equation}

with a function $\lambda: \mathbb{R}\rightarrow\mathbb{R}$ subject to constraints $\lambda(0)=0$ and $\lambda'(0)=1$. We define

\begin{equation}\label{formula_M}
M(t,x;T) = \log E \left( e^{-\lambda(T)(X(T)-x)} \middle | X(t)=x \right)
\end{equation}

which is central to the computation of numeraire and zerobond values in the model. The transition density is given in \cite{betaeta}, (4.6a), (4.6b) and (4.8). For $\eta < 0.5$ a reflecting barrier at $x=-1/\beta$ can be specified in \ref{dynamics}, the corresponding amendments in the density are also given in \cite{betaeta}, p. 241, last paragraph.

For the special case of $\eta=0$ a closed form representation for the density (C.3) and for $M(t,x;T)$ (C.5) is given, but only for the case of no barrier. For $\eta=0.5$ a closed form expression for $M(t,x;T)$ is given in (4.15). For $\eta=1$ we have (4.10b) for the density (however there is no closed form expression for $M(t,x;T)$ in this case).

We follow \cite{piterbarg} and describe $\lambda$ by a constant mean reversion $\kappa \neq 0$ via

\begin{equation}
\lambda = \frac{1 - e^{-\kappa t}}{\kappa}
\end{equation}

The normalization constraints $\lambda(0)=0, \lambda'(0)=1$ are immediately verified. Furthermore

\begin{equation}
\kappa = -\frac{\lambda''(t)}{\lambda'(t)}
\end{equation}

Note that $\lambda > 0$ by construction.

\section{Numerical Issues}

\subsection{Computation of $p(t,x;\overline{t},\overline{x})$}

The formulas for the density (\cite{betaeta}, (4.6a), (4.6b) and (4.8)) contain expressions

\begin{equation}\label{bessel}
I_\nu(y\overline{y}/(\overline{\tau}-\tau)) e^{-(\overline{y}^2+y^2)/(2(\overline{\tau}-\tau))}
\end{equation}

with $I_\nu$ the modified Bessel function of the first kind (likewise $I_{-\nu}$ and $K_\nu$ appear at other places). We can rewrite \ref{bessel} as

\begin{equation}
I_\nu(y\overline{y}/(\overline{\tau}-\tau))e^{-y\overline{y}/(\overline{\tau}-\tau)} e^{-(\overline{y}+y)^2/(2(\overline{\tau}-\tau))}
\end{equation}

so that we can use the exponentially weighted implementation of the modified Bessel function to cover the first two factors. This is necessary for a numerically stable result.

For values $\eta$ close to, but not equal to $1$ the closed form representation from the paper gets numerically unstable. Therefore we interpolate the density between some threshold value $\eta_M$ and $\eta=1$ linearly. The threshold value is set to the largest tabulated value for $\eta$ (see below, this is typically near $0.99$), since this is guaranteed to deliver stable values.


\subsection{Tabulation of $M(t,x,T)$}

For $\eta=0.5$ we can use a closed form expression for $M$, see above. For $\eta=1$ we can compute

\begin{equation}\label{M_eta_1}
M(t,x;T) = \int_{-\infty}^\infty e^{-\lambda(T)\beta^{-1} (\exp(\beta \overline{y}) - \exp(\beta y))} e^{-z^2} dz
\end{equation}

using a Gauss Hermite Scheme (we use $8$ points which already seem to give enough accuracy). Here we write $\overline{y} = z\sqrt{2v} + y - \beta v / 2$, $v=\overline{\tau}-\tau$ and use the transformation from \cite{betaeta}, (4.4)

\begin{equation}
y = \begin{cases}
\frac{|1+\beta x|^{1-\eta}}{\beta(1-\eta)} & \eta\neq 1 \\
\log(1+\beta x)/\beta & \eta=1 \\
\end{cases}
\end{equation}

Furthermore we write 

\begin{equation}
\tau = \int_0^{t} \alpha(s) ds
\end{equation}

Overlined variables like $\overline{y}$ and $\overline{\tau}$ denote the quantities w.r.t. $T$ (or $\overline{t}$) instead of $t$. In the following we look at the case $\eta < 1$ (and $\eta \neq 0.5$, although the computations would work for this very case as well).

The transition density is originally expressed in variables $y, \overline{y}$ and $\tau, \overline{\tau}$ instead of $x, \overline{x}$ and $t, \overline{t}$. It is obvious from \cite{betaeta}, (4.6a), (4.6b), (4.7) and (4.8) that $p$ can be written as a function of $\overline{\tau}-\tau$. Furthermore for any $a > 0$ we have the homogenity relation

\begin{equation}\label{phom}
p\left(a^2(\overline{\tau}-\tau), ay, a\overline{y}\right) = a^{1/(\eta-1)} p\left(\overline{\tau}-\tau, y, \overline{y}\right)
\end{equation}

We have to compute

\begin{equation}
M(t,x;\overline{t}) = \log \int_{-\infty}^{\infty} p(t,x;\overline{t},\overline{x}) e^{-\lambda(\overline{t})(\overline{x}-x)}d\overline{x}
\end{equation}

The integral can be rewritten using variable transformation and \ref{phom} as

\begin{equation}
\int_{0}^{\infty} p^* ( \lambda^{2-2\eta} \beta^{2\eta} (1-\eta)^{2\eta} v, u^{1-\eta}(1-\eta)^{\eta-1}, \overline{u}^{1-\eta}(1-\eta)^{\eta-1} ) e^{-(\overline{u}-u)} d\overline{u}
\end{equation}

where $p^*= p\cdot [(1-\eta)\beta]^{-\eta/(\eta-1)}$ with $u = \lambda(\overline{t})\beta^{-1}|1+\beta x|$. Introducing $S:=\beta^2v/(1+\beta x)^{2-2\eta}$ this reads

\begin{equation}
\int_{0}^{\infty} p^* ( S (1-\eta)^{2\eta} u^{2-2\eta}, u^{1-\eta}(1-\eta)^{\eta-1}, \overline{u}^{1-\eta}(1-\eta)^{\eta-1} ) e^{-(\overline{u}-u)} d\overline{u}
\end{equation}

which can be tabulated. Actually we tabulate $M$ in $(S^*,u)$ instead of $(S,u)$ with

\begin{equation}
S^* = S u^{2-\eta/2}
\end{equation}

since this allows to cover more ``useful'' values by a rectangular set for $(S^*,u)$.

We tabulate $[10^{-4},20] \times [10^{-4},1000]$ with $100$ grid points in each direction with a density of $10^{-4}$ and concentrating point at $(10^{-4}, 10^{-4})$ (in the sense of QuantLib's concentrating mesher). In $\eta$ direction we use $100$ grid points as well (with a concentrating point at $0.5$ and density $1.0$). The interpolation is done linearly. Since for $\eta=1$ we do not have a tabulation we interpolation $M$ between the largest tabulated point for $\eta$ and $\eta=1$, for the latter we use \ref{M_eta_1}.

\subsection{Tabulation of $P(y = 0)$ and cutoff of small values}

For $\eta \geq 0.5$ there is a closed form expression for the probability of $y$ being zero given in \cite{betaeta}, (4.8). The paper does not give such an expression for $\eta<0.5$ when a reflecting barrier condition at $y=0$ is chosen. In any case we need a tabulation of these values because the evaluation of the incomplete upper gamma function $\Gamma(\cdot,\cdot)$ slows down the pricing considerably if implemented explicitly.

The tabulation is straightforward. For $\eta$ we use the same grid as for the tabulation of $M$. For $y$ and $v$ we tabulate $50 \times 50$ points and cover $[10^{-4},10] x [10^{-4},1]$ with a density of $10^{-4}$ and concentrating point $(10^{-4},10^{-4})$.

Since when computing $M$ very huge terms are multiplied with very small values of $P(y=0)$ we choose to set this probability to zero in case it is below a threshold (of $10^{-6}$).

\section{Examples}

\subsection{Yield-state anomaly}
\label{ys-anomaly}

For larger values of both $\beta$ and $\alpha$ we observe an anomaly in the functional dependency of the zero yield on the model's state variable. The usual expectation would be that the zero yield is monotone ascending in the state variable. For small $\beta$ this can indeed be verified, see figure \ref{zeroyield_beta01} for an example with $\beta=0.1$ (the reversion $\kappa$ is $1\%$, the yield term structure flat at $3\%$ in this example).

\begin{figure}[ht]
\caption{10y zero yield at $t=10$ conditional on $x(10)$ for different levels of $\alpha$, $\beta=0.1$}
\label{zeroyield_beta01}
\begin{gnuplot}[scale=1,terminal=epslatex,terminaloptions=color] 
set xrange [*:*]
set yrange [*:*]
set zrange [*:*]
set xlabel "alpha"
set ylabel "x(10)"
set zlabel "r(20,10,x(10))" offset -5,0
set xtics 0.02
set ytics 0.25
set ztics 0.5
set contour
set cntrparam levels 25
unset key
set grid
set view 52,70
splot 'beta_01.dat' u 1:2:3:3 w l palette
\end{gnuplot}
\end{figure}

For higher values of $\beta$ however, the dependency of the zero yield on the state flattens and even reverses, so that for high volatilities $\alpha$ the zero yield is monotone \textit{descending} in the state. Obviously this can produce unintuitive results like swaption prices that are not monotone ascending w.r.t. the model volatility. Figure \ref{zeroyield_beta50} shows an example in this direction.

\begin{figure}[ht]
\caption{10y zero yield at $t=10$ conditional on $x(10)$ for different levels of $\alpha$, $\beta=5$}
\label{zeroyield_beta50}
\begin{gnuplot}[scale=1,terminal=epslatex,terminaloptions=color] 
set xrange [*:*]
set yrange [-0.2:*]
set zrange [*:*]
set xlabel "alpha"
set ylabel "x(t)"
set zlabel "r(20,10,x(10))" offset -5,0
set xtics 0.02
set ytics 0.25
set ztics 0.5
set contour
set cntrparam levels 25
unset key
set grid
set view 52,70
splot 'beta_50.dat' u 1:2:3:3 w l palette
\end{gnuplot}
\end{figure}

\subsection{Hull White case}

For $\eta=0, \kappa=0$ according to \cite{betaeta}, C1 we can expect to replicate the Hull White model. Table \ref{hwcase} shows the pricing of 10y into 10y swaptions in the GSR and $\beta-\eta$ model. We use the respective integral engines with $64$ integration points covering $7$ standard deviations of the state variable. The yield term structure is flat at $3\%$ and we price swaptions with strikes

\begin{enumerate}
\item 0\%, atm-200bp, atm-150bp, atm-100bp, atm-50bp, atm-25bp (Receiver)
\item atm+25bp, atm+50bp, atm+100bp, atm+150bp, atm+200bp (Payer)
\end{enumerate}

Table \ref{hwcase} shows the results. GSR and $\beta-\eta$ pricings are consistent. Also the usage of tabulated values does not impact the pricing accuracy. The parity error is defined as

\begin{equation}
\frac{\pi - \rho}{A} - (f-k)
\end{equation}

with $\pi$ and $\rho$ the payer resp. receiver swaption price, $A$ the annuity, $f$ the forward swap rate and $k$ the strike, where $f$ is computed on the inital rate curve outside the model as the reference value. In general we find the parity error to be a good measure for both the error arising from numerics and approximations from the tabulation.

\begin{table}[ht]
\caption{Swaption pricing check against GSR for $\eta=0$, $\beta=0.1$, $\sigma=0.0075$, $\kappa=0$ using tabulated values and full integration}
\begin{tabular}{r | r | r | r | r | r}
strike & npv GSR & npv $\beta\eta$ tab & npv $\beta\eta$ full & parity error tab & parity error full \\ \hline
0 & 0.0073954 & 0.0073943 & 0.0073943 & 6.3182e-08 & 6.2763e-08 \\
0.010472 & 0.017449 & 0.017447 & 0.017447 & 7.5887e-08 & 7.5468e-08 \\
0.015472 & 0.024959 & 0.024957 & 0.024957 & -5.7966e-08 & -5.8384e-08 \\
0.020472 & 0.034625 & 0.034623 & 0.034623 & -1.5272e-07 & -1.5313e-07 \\
0.025472 & 0.04672 & 0.046729 & 0.046729 & 2.1575e-08 & 2.1157e-08 \\
0.027972 & 0.053726 & 0.053734 & 0.053734 & 1.2001e-07 & 1.196e-07 \\
0.030472 & 0.061376 & 0.06138 & 0.06138 & -1.9706e-07 & -1.9665e-07 \\
0.032972 & 0.053905 & 0.053909 & 0.053909 & -2.2308e-07 & -2.2267e-07 \\
0.035472 & 0.04707 & 0.047073 & 0.047073 & -2.4932e-07 & -2.4891e-07 \\
0.040472 & 0.035247 & 0.035248 & 0.035248 & -2.9785e-07 & -2.9743e-07 \\
0.045472 & 0.025739 & 0.025737 & 0.025737 & -3.3749e-07 & -3.3707e-07 \\
0.050472 & 0.018306 & 0.018303 & 0.018303 & -3.6517e-07 & -3.6475e-07
\end{tabular}
\label{hwcase}
\end{table}

\subsection{Influence of $\beta$, $\eta$ and $\kappa$ on the smile}

We continue to examine the influence of the model parameters $\beta$, $\eta$ and $\kappa$ on the implied smile. We look at reversion levels $\kappa \in { -0.05, -0.01, 0, 0.01 ,0.05 }$. The exponent $\eta$ is chosen as $0$, $0.2$, $0.5$, $0.8$ and $1.0$ respectively. 

The range for $\beta$ is set to $0.1$, $1.0$, $2.0$, $3.0$, $4.0$, $5.0$ here. Note that too large values can cause difficult conditions in the model as shows in \ref{ys-anomaly}. Actually in one of the following examples the calibration of the model to an atm swaption with $16\%$ market volatility already fails for $\beta=5$ (namely for $\eta=1$, $\kappa=0.05$) due to the described issue. Hagan himself does not give a hint to which values for $\beta$ are reasonable on the other hand.

Figures \ref{smile_02_00}, \ref{smile_05_00}, \ref{smile_08_m01} and \ref{smile_10_05} show the dependency of the smile of the different values of $\beta$ for fixed $\eta$ and $\kappa$. In general for higher $\eta$ and higher $\kappa$ the dependency on $beta$ gets stronger and higher $\beta$ decrease the skew. For these example cases we recalibrate the model volatility for each $(\eta,\beta,\kappa)$ such that an atm swaption with $16\%$ implied lognormal volatility is matched.

Table \ref{skewtable} lists all example cases (except the one where the calibration failes, see above) together with the implied volatilities for atm, atm-200bp, atm+200bp and the skew and curvature of the smile for atm+-200bp, defined as

\begin{eqnarray}
\text{skew} & := & \sigma_{\text{atm}+200\text{bp}} - \sigma_{\text{atm}-200\text{bp}} \\
\text{curvature} & := & \sigma_{\text{atm}+200\text{bp}} -2\sigma_{\text{atm}} + \sigma_{\text{atm}-200\text{bp}}
\end{eqnarray}

\begin{figure}[ht]
\caption{implied lognormal volatility smiles, $\eta=0.2$, $\kappa=0.0$}
\label{smile_02_00}
\begin{gnuplot}[scale=1,terminal=epslatex,terminaloptions=color] 
eta = 0.2
kappa = 0.0
set key
set grid
set xlabel "strike"
set ylabel "implied lognormal volatility"
plot "smiles.dat" u ($1==eta && $3==kappa && $2==0.1 && $4>0 ? $4:1/0):7 w l title "beta = 0.1", "" u ($1==eta && $3==kappa && $2==1.0 && $4>0 ? $4:1/0):7 w l title "beta = 1.0", "" u ($1==eta && $3==kappa && $2==2.0 && $4>0 ? $4:1/0):7 w l title "beta = 2.0", "" u ($1==eta && $3==kappa && $2==3.0 && $4>0 ? $4:1/0):7 w l title "beta = 3.0", "" u ($1==eta && $3==kappa && $2==4.0 && $4>0 ? $4:1/0):7 w l title "beta = 4.0", "" u ($1==eta && $3==kappa && $2==5.0 && $4>0 ? $4:1/0):7 w l lc 1 title "beta = 5.0"
\end{gnuplot}
\end{figure}

\begin{figure}[ht]
\caption{implied lognormal volatility smiles, $\eta=0.5$, $\kappa=0.0$}
\label{smile_05_00}
\begin{gnuplot}[scale=1,terminal=epslatex,terminaloptions=color] 
eta = 0.5
kappa = 0.0
set key
set grid
set xlabel "strike"
set ylabel "implied lognormal volatility"
plot "smiles.dat" u ($1==eta && $3==kappa && $2==0.1 && $4>0 ? $4:1/0):7 w l title "beta = 0.1", "" u ($1==eta && $3==kappa && $2==1.0 && $4>0 ? $4:1/0):7 w l title "beta = 1.0", "" u ($1==eta && $3==kappa && $2==2.0 && $4>0 ? $4:1/0):7 w l title "beta = 2.0", "" u ($1==eta && $3==kappa && $2==3.0 && $4>0 ? $4:1/0):7 w l title "beta = 3.0", "" u ($1==eta && $3==kappa && $2==4.0 && $4>0 ? $4:1/0):7 w l title "beta = 4.0", "" u ($1==eta && $3==kappa && $2==5.0 && $4>0 ? $4:1/0):7 w l lc 1 title "beta = 5.0"
\end{gnuplot}
\end{figure}

\begin{figure}[ht]
\caption{implied lognormal volatility smiles, $\eta=0.8$, $\kappa=-0.01$}
\label{smile_08_m01}
\begin{gnuplot}[scale=1,terminal=epslatex,terminaloptions=color] 
eta = 0.8
kappa = -0.01
set key
set grid
set xlabel "strike"
set ylabel "implied lognormal volatility"
plot "smiles.dat" u ($1==eta && $3==kappa && $2==0.1 && $4>0 ? $4:1/0):7 w l title "beta = 0.1", "" u ($1==eta && $3==kappa && $2==1.0 && $4>0 ? $4:1/0):7 w l title "beta = 1.0", "" u ($1==eta && $3==kappa && $2==2.0 && $4>0 ? $4:1/0):7 w l title "beta = 2.0", "" u ($1==eta && $3==kappa && $2==3.0 && $4>0 ? $4:1/0):7 w l title "beta = 3.0", "" u ($1==eta && $3==kappa && $2==4.0 && $4>0 ? $4:1/0):7 w l title "beta = 4.0", "" u ($1==eta && $3==kappa && $2==5.0 && $4>0 ? $4:1/0):7 w l lc 1 title "beta = 5.0"
\end{gnuplot}
\end{figure}

\begin{figure}[ht]
\caption{implied lognormal volatility smiles, $\eta=0.8$, $\kappa=0.01$}
\label{smile_08_01}
\begin{gnuplot}[scale=1,terminal=epslatex,terminaloptions=color] 
eta = 0.8
kappa = 0.01
set key
set grid
set xlabel "strike"
set ylabel "implied lognormal volatility"
plot "smiles.dat" u ($1==eta && $3==kappa && $2==0.1 && $4>0 ? $4:1/0):7 w l title "beta = 0.1", "" u ($1==eta && $3==kappa && $2==1.0 && $4>0 ? $4:1/0):7 w l title "beta = 1.0", "" u ($1==eta && $3==kappa && $2==2.0 && $4>0 ? $4:1/0):7 w l title "beta = 2.0", "" u ($1==eta && $3==kappa && $2==3.0 && $4>0 ? $4:1/0):7 w l title "beta = 3.0", "" u ($1==eta && $3==kappa && $2==4.0 && $4>0 ? $4:1/0):7 w l title "beta = 4.0", "" u ($1==eta && $3==kappa && $2==5.0 && $4>0 ? $4:1/0):7 w l lc 1 title "beta = 5.0"
\end{gnuplot}
\end{figure}

\begin{figure}[ht]
\caption{implied lognormal volatility smiles, $\eta=1.0$, $\kappa=0.05$}
\label{smile_10_05}
\begin{gnuplot}[scale=1,terminal=epslatex,terminaloptions=color] 
eta = 1.0
kappa = 0.05
set key
set grid
set xlabel "strike"
set ylabel "implied lognormal volatility"
plot "smiles.dat" u ($1==eta && $3==kappa && $2==0.1 && $4>0 ? $4:1/0):7 w l title "beta = 0.1", "" u ($1==eta && $3==kappa && $2==1.0 && $4>0 ? $4:1/0):7 w l title "beta = 1.0", "" u ($1==eta && $3==kappa && $2==2.0 && $4>0 ? $4:1/0):7 w l title "beta = 2.0", "" u ($1==eta && $3==kappa && $2==3.0 && $4>0 ? $4:1/0):7 w l title "beta = 3.0", "" u ($1==eta && $3==kappa && $2==4.0 && $4>0 ? $4:1/0):7 w l title "beta = 4.0"
\end{gnuplot}
\end{figure}

\begin{longtable}{r | r | r | r | r | r | r | r}
\caption{Generated skew and curvature} \\
$\eta$ & $\kappa$ & $\beta$ & $-200$ & atm & $+200$ & skew & curvature \\ \hline
\endfirsthead
$\eta$ & $\kappa$ & $\beta$ & $-200$ & atm & $+200$ & skew & curvature \\ \hline
\endhead
\multicolumn{8}{r}{\textit{Continued on next page ...}} \\
\endfoot
\endlastfoot
0 & -0.05 & 0.1 & 0.2648 & 0.1601 & 0.1227 & -0.1421 & 0.06737 \\
0 & -0.05 & 1 & 0.2648 & 0.1601 & 0.1227 & -0.1421 & 0.06737 \\
0 & -0.05 & 2 & 0.2648 & 0.1601 & 0.1227 & -0.1421 & 0.06737 \\
0 & -0.05 & 3 & 0.2648 & 0.1601 & 0.1227 & -0.1421 & 0.06737 \\
0 & -0.05 & 4 & 0.2648 & 0.1601 & 0.1227 & -0.1421 & 0.06737 \\
0 & -0.05 & 5 & 0.2648 & 0.1601 & 0.1227 & -0.1421 & 0.06737 \\ \hline
0 & -0.01 & 0.1 & 0.2628 & 0.16 & 0.1234 & -0.1394 & 0.06615 \\
0 & -0.01 & 1 & 0.2628 & 0.16 & 0.1234 & -0.1394 & 0.06615 \\
0 & -0.01 & 2 & 0.2628 & 0.16 & 0.1234 & -0.1394 & 0.06615 \\
0 & -0.01 & 3 & 0.2628 & 0.16 & 0.1234 & -0.1394 & 0.06615 \\
0 & -0.01 & 4 & 0.2628 & 0.16 & 0.1234 & -0.1394 & 0.06615 \\
0 & -0.01 & 5 & 0.2628 & 0.16 & 0.1234 & -0.1394 & 0.06615 \\ \hline
0 & 0 & 0.1 & 0.2626 & 0.16 & 0.1238 & -0.1388 & 0.06631 \\
0 & 0 & 1 & 0.2626 & 0.16 & 0.1238 & -0.1388 & 0.06631 \\
0 & 0 & 2 & 0.2626 & 0.16 & 0.1238 & -0.1388 & 0.06631 \\
0 & 0 & 3 & 0.2626 & 0.16 & 0.1238 & -0.1388 & 0.06631 \\
0 & 0 & 4 & 0.2626 & 0.16 & 0.1238 & -0.1388 & 0.06631 \\
0 & 0 & 5 & 0.2626 & 0.16 & 0.1238 & -0.1388 & 0.06631 \\ \hline
0 & 0.01 & 0.1 & 0.2621 & 0.16 & 0.1239 & -0.1381 & 0.066 \\
0 & 0.01 & 1 & 0.2621 & 0.16 & 0.1239 & -0.1381 & 0.066 \\
0 & 0.01 & 2 & 0.2621 & 0.16 & 0.1239 & -0.1381 & 0.066 \\
0 & 0.01 & 3 & 0.2621 & 0.16 & 0.1239 & -0.1381 & 0.066 \\
0 & 0.01 & 4 & 0.2621 & 0.16 & 0.1239 & -0.1381 & 0.066 \\
0 & 0.01 & 5 & 0.2621 & 0.16 & 0.1239 & -0.1381 & 0.066 \\ \hline
0 & 0.05 & 0.1 & 0.2599 & 0.1599 & 0.1245 & -0.1354 & 0.06463 \\
0 & 0.05 & 1 & 0.2599 & 0.1599 & 0.1245 & -0.1354 & 0.06463 \\
0 & 0.05 & 2 & 0.2599 & 0.1599 & 0.1245 & -0.1354 & 0.06463 \\
0 & 0.05 & 3 & 0.2599 & 0.1599 & 0.1245 & -0.1354 & 0.06463 \\
0 & 0.05 & 4 & 0.2599 & 0.1599 & 0.1245 & -0.1354 & 0.06463 \\
0 & 0.05 & 5 & 0.2601 & 0.1599 & 0.1246 & -0.1356 & 0.06488 \\ \hline
0.2 & -0.05 & 0.1 & 0.2648 & 0.1601 & 0.1228 & -0.142 & 0.06736 \\
0.2 & -0.05 & 1 & 0.2646 & 0.1601 & 0.1229 & -0.1417 & 0.06723 \\
0.2 & -0.05 & 2 & 0.2643 & 0.1601 & 0.123 & -0.1413 & 0.06708 \\
0.2 & -0.05 & 3 & 0.264 & 0.1601 & 0.1231 & -0.1409 & 0.06692 \\
0.2 & -0.05 & 4 & 0.2637 & 0.1601 & 0.1232 & -0.1405 & 0.06677 \\
0.2 & -0.05 & 5 & 0.2635 & 0.1601 & 0.1233 & -0.1401 & 0.0666 \\ \hline
0.2 & -0.01 & 0.1 & 0.2627 & 0.16 & 0.1234 & -0.1393 & 0.06612 \\
0.2 & -0.01 & 1 & 0.2623 & 0.16 & 0.1236 & -0.1387 & 0.06588 \\
0.2 & -0.01 & 2 & 0.2618 & 0.16 & 0.1239 & -0.1379 & 0.06561 \\
0.2 & -0.01 & 3 & 0.2613 & 0.16 & 0.1241 & -0.1372 & 0.06534 \\
0.2 & -0.01 & 4 & 0.2608 & 0.16 & 0.1243 & -0.1365 & 0.06505 \\
0.2 & -0.01 & 5 & 0.2602 & 0.16 & 0.1245 & -0.1357 & 0.06475 \\ \hline
0.2 & 0 & 0.1 & 0.2625 & 0.16 & 0.1238 & -0.1387 & 0.06627 \\
0.2 & 0 & 1 & 0.262 & 0.16 & 0.124 & -0.1379 & 0.06598 \\
0.2 & 0 & 2 & 0.2614 & 0.16 & 0.1243 & -0.1371 & 0.06563 \\
0.2 & 0 & 3 & 0.2607 & 0.16 & 0.1245 & -0.1362 & 0.06527 \\
0.2 & 0 & 4 & 0.2601 & 0.16 & 0.1248 & -0.1353 & 0.06488 \\
0.2 & 0 & 5 & 0.2594 & 0.16 & 0.1251 & -0.1343 & 0.06448 \\ \hline
0.2 & 0.01 & 0.1 & 0.262 & 0.16 & 0.124 & -0.138 & 0.06595 \\
0.2 & 0.01 & 1 & 0.2614 & 0.16 & 0.1242 & -0.1372 & 0.06561 \\
0.2 & 0.01 & 2 & 0.2607 & 0.16 & 0.1245 & -0.1362 & 0.0652 \\
0.2 & 0.01 & 3 & 0.2599 & 0.16 & 0.1248 & -0.1351 & 0.06477 \\
0.2 & 0.01 & 4 & 0.2592 & 0.16 & 0.1251 & -0.134 & 0.06431 \\
0.2 & 0.01 & 5 & 0.2584 & 0.16 & 0.1254 & -0.1329 & 0.06383 \\ \hline
0.2 & 0.05 & 0.1 & 0.2598 & 0.1599 & 0.1246 & -0.1352 & 0.06456 \\
0.2 & 0.05 & 1 & 0.2587 & 0.1599 & 0.1251 & -0.1336 & 0.06395 \\
0.2 & 0.05 & 2 & 0.2573 & 0.1599 & 0.1257 & -0.1316 & 0.06318 \\
0.2 & 0.05 & 3 & 0.2557 & 0.1599 & 0.1263 & -0.1294 & 0.06226 \\
0.2 & 0.05 & 4 & 0.254 & 0.1599 & 0.127 & -0.1271 & 0.06126 \\
0.2 & 0.05 & 5 & 0.252 & 0.1599 & 0.1276 & -0.1244 & 0.0598 \\ \hline
0.5 & -0.05 & 0.1 & 0.2648 & 0.1601 & 0.1228 & -0.142 & 0.06734 \\
0.5 & -0.05 & 1 & 0.2642 & 0.1601 & 0.123 & -0.1411 & 0.067 \\
0.5 & -0.05 & 2 & 0.2635 & 0.1601 & 0.1233 & -0.1402 & 0.06663 \\
0.5 & -0.05 & 3 & 0.2628 & 0.1601 & 0.1236 & -0.1392 & 0.06624 \\
0.5 & -0.05 & 4 & 0.2621 & 0.1601 & 0.1239 & -0.1382 & 0.06585 \\
0.5 & -0.05 & 5 & 0.2615 & 0.1601 & 0.1242 & -0.1373 & 0.06545 \\ \hline
0.5 & -0.01 & 0.1 & 0.2627 & 0.16 & 0.1235 & -0.1392 & 0.06609 \\
0.5 & -0.01 & 1 & 0.2616 & 0.16 & 0.124 & -0.1376 & 0.06551 \\
0.5 & -0.01 & 2 & 0.2604 & 0.16 & 0.1245 & -0.1358 & 0.06486 \\
0.5 & -0.01 & 3 & 0.2591 & 0.16 & 0.1251 & -0.134 & 0.06418 \\
0.5 & -0.01 & 4 & 0.2579 & 0.16 & 0.1257 & -0.1322 & 0.06348 \\
0.5 & -0.01 & 5 & 0.2566 & 0.16 & 0.1262 & -0.1304 & 0.06275 \\ \hline
0.5 & 0 & 0.1 & 0.2624 & 0.16 & 0.1238 & -0.1386 & 0.06623 \\
0.5 & 0 & 1 & 0.2611 & 0.16 & 0.1244 & -0.1367 & 0.06548 \\
0.5 & 0 & 2 & 0.2596 & 0.16 & 0.125 & -0.1346 & 0.06462 \\
0.5 & 0 & 3 & 0.2581 & 0.16 & 0.1257 & -0.1324 & 0.06372 \\
0.5 & 0 & 4 & 0.2565 & 0.16 & 0.1263 & -0.1301 & 0.06278 \\
0.5 & 0 & 5 & 0.2549 & 0.16 & 0.127 & -0.1279 & 0.06185 \\ \hline
0.5 & 0.01 & 0.1 & 0.2619 & 0.16 & 0.124 & -0.1379 & 0.0659 \\
0.5 & 0.01 & 1 & 0.2604 & 0.16 & 0.1247 & -0.1357 & 0.06505 \\
0.5 & 0.01 & 2 & 0.2586 & 0.16 & 0.1254 & -0.1332 & 0.06404 \\
0.5 & 0.01 & 3 & 0.2568 & 0.16 & 0.1261 & -0.1307 & 0.06298 \\
0.5 & 0.01 & 4 & 0.255 & 0.16 & 0.1269 & -0.1281 & 0.06186 \\
0.5 & 0.01 & 5 & 0.253 & 0.16 & 0.1276 & -0.1254 & 0.06068 \\ \hline
0.5 & 0.05 & 0.1 & 0.2596 & 0.1599 & 0.1247 & -0.135 & 0.06447 \\
0.5 & 0.05 & 1 & 0.2568 & 0.1599 & 0.1259 & -0.1309 & 0.06295 \\
0.5 & 0.05 & 2 & 0.2536 & 0.1599 & 0.1274 & -0.1262 & 0.06121 \\
0.5 & 0.05 & 3 & 0.2502 & 0.1599 & 0.1289 & -0.1213 & 0.05926 \\
0.5 & 0.05 & 4 & 0.2462 & 0.1599 & 0.1304 & -0.1158 & 0.05684 \\
0.5 & 0.05 & 5 & 0.2416 & 0.1599 & 0.1319 & -0.1097 & 0.05373 \\ \hline
0.8 & -0.05 & 0.1 & 0.2647 & 0.1601 & 0.1228 & -0.1419 & 0.06733 \\
0.8 & -0.05 & 1 & 0.2638 & 0.1601 & 0.1232 & -0.1406 & 0.0668 \\
0.8 & -0.05 & 2 & 0.2627 & 0.1601 & 0.1237 & -0.139 & 0.0662 \\
0.8 & -0.05 & 3 & 0.2616 & 0.1601 & 0.1241 & -0.1375 & 0.06559 \\
0.8 & -0.05 & 4 & 0.2605 & 0.1601 & 0.1246 & -0.136 & 0.06498 \\
0.8 & -0.05 & 5 & 0.2595 & 0.1601 & 0.125 & -0.1344 & 0.06434 \\ \hline
0.8 & -0.01 & 0.1 & 0.2626 & 0.16 & 0.1235 & -0.1391 & 0.06604 \\
0.8 & -0.01 & 1 & 0.2609 & 0.16 & 0.1243 & -0.1366 & 0.06513 \\
0.8 & -0.01 & 2 & 0.259 & 0.16 & 0.1252 & -0.1338 & 0.0641 \\
0.8 & -0.01 & 3 & 0.257 & 0.16 & 0.1261 & -0.1309 & 0.06306 \\
0.8 & -0.01 & 4 & 0.255 & 0.16 & 0.127 & -0.1281 & 0.06198 \\
0.8 & -0.01 & 5 & 0.253 & 0.16 & 0.1278 & -0.1252 & 0.06088 \\ \hline
0.8 & 0 & 0.1 & 0.2623 & 0.16 & 0.1239 & -0.1385 & 0.06619 \\
0.8 & 0 & 1 & 0.2602 & 0.16 & 0.1248 & -0.1354 & 0.065 \\
0.8 & 0 & 2 & 0.2579 & 0.16 & 0.1258 & -0.132 & 0.06362 \\
0.8 & 0 & 3 & 0.2555 & 0.16 & 0.1268 & -0.1287 & 0.06227 \\
0.8 & 0 & 4 & 0.2531 & 0.16 & 0.1278 & -0.1253 & 0.06092 \\
0.8 & 0 & 5 & 0.2507 & 0.16 & 0.1288 & -0.1219 & 0.05957 \\ \hline
0.8 & 0.01 & 0.1 & 0.2618 & 0.16 & 0.124 & -0.1377 & 0.06584 \\
0.8 & 0.01 & 1 & 0.2594 & 0.16 & 0.1251 & -0.1343 & 0.06448 \\
0.8 & 0.01 & 2 & 0.2566 & 0.16 & 0.1263 & -0.1304 & 0.06288 \\
0.8 & 0.01 & 3 & 0.2538 & 0.16 & 0.1274 & -0.1264 & 0.06124 \\
0.8 & 0.01 & 4 & 0.2509 & 0.16 & 0.1286 & -0.1224 & 0.05948 \\
0.8 & 0.01 & 5 & 0.2479 & 0.16 & 0.1297 & -0.1182 & 0.05765 \\ \hline
0.8 & 0.05 & 0.1 & 0.2594 & 0.1599 & 0.1247 & -0.1347 & 0.06439 \\
0.8 & 0.05 & 1 & 0.2551 & 0.1599 & 0.1267 & -0.1284 & 0.0621 \\
0.8 & 0.05 & 2 & 0.2503 & 0.1598 & 0.129 & -0.1213 & 0.05957 \\
0.8 & 0.05 & 3 & 0.2448 & 0.1598 & 0.1311 & -0.1138 & 0.05623 \\
0.8 & 0.05 & 4 & 0.2387 & 0.1597 & 0.1332 & -0.1055 & 0.05244 \\
0.8 & 0.05 & 5 & 0.2317 & 0.1597 & 0.1358 & -0.09582 & 0.04799 \\ \hline
1 & -0.05 & 0.1 & 0.2647 & 0.1601 & 0.1228 & -0.1419 & 0.0673 \\
1 & -0.05 & 1 & 0.2635 & 0.1601 & 0.1233 & -0.1402 & 0.06664 \\
1 & -0.05 & 2 & 0.2622 & 0.1601 & 0.1239 & -0.1383 & 0.0659 \\
1 & -0.05 & 3 & 0.2609 & 0.1601 & 0.1245 & -0.1364 & 0.06515 \\
1 & -0.05 & 4 & 0.2595 & 0.1601 & 0.125 & -0.1345 & 0.06439 \\
1 & -0.05 & 5 & 0.2582 & 0.1601 & 0.1256 & -0.1326 & 0.06362 \\ \hline
1 & -0.01 & 0.1 & 0.2625 & 0.16 & 0.1235 & -0.139 & 0.06602 \\
1 & -0.01 & 1 & 0.2604 & 0.16 & 0.1245 & -0.1359 & 0.0649 \\
1 & -0.01 & 2 & 0.2581 & 0.16 & 0.1256 & -0.1324 & 0.06364 \\
1 & -0.01 & 3 & 0.2557 & 0.16 & 0.1267 & -0.129 & 0.06236 \\
1 & -0.01 & 4 & 0.2533 & 0.16 & 0.1278 & -0.1255 & 0.06104 \\
1 & -0.01 & 5 & 0.2508 & 0.16 & 0.1289 & -0.1219 & 0.05971 \\ \hline
1 & 0 & 0.1 & 0.2623 & 0.16 & 0.1239 & -0.1384 & 0.06615 \\
1 & 0 & 1 & 0.2597 & 0.16 & 0.125 & -0.1346 & 0.06468 \\
1 & 0 & 2 & 0.2567 & 0.16 & 0.1263 & -0.1305 & 0.063 \\
1 & 0 & 3 & 0.2539 & 0.16 & 0.1275 & -0.1264 & 0.0614 \\
1 & 0 & 4 & 0.251 & 0.16 & 0.1287 & -0.1223 & 0.05978 \\
1 & 0 & 5 & 0.2481 & 0.16 & 0.1299 & -0.1182 & 0.05811 \\ \hline
1 & 0.01 & 0.1 & 0.2617 & 0.16 & 0.1241 & -0.1376 & 0.06581 \\
1 & 0.01 & 1 & 0.2587 & 0.16 & 0.1254 & -0.1334 & 0.06412 \\
1 & 0.01 & 2 & 0.2553 & 0.16 & 0.1268 & -0.1285 & 0.06217 \\
1 & 0.01 & 3 & 0.2519 & 0.16 & 0.1282 & -0.1237 & 0.06013 \\
1 & 0.01 & 4 & 0.2483 & 0.16 & 0.1296 & -0.1187 & 0.05799 \\
1 & 0.01 & 5 & 0.2449 & 0.16 & 0.131 & -0.1139 & 0.05591 \\ \hline
1 & 0.05 & 0.1 & 0.2593 & 0.1599 & 0.1248 & -0.1345 & 0.06431 \\
1 & 0.05 & 1 & 0.2541 & 0.1599 & 0.1273 & -0.1268 & 0.06154 \\
1 & 0.05 & 2 & 0.2481 & 0.1599 & 0.1299 & -0.1182 & 0.05828 \\
1 & 0.05 & 3 & 0.2416 & 0.1599 & 0.1323 & -0.1093 & 0.05422 \\
1 & 0.05 & 4 & 0.2344 & 0.1598 & 0.135 & -0.0994 & 0.0498
\label{skewtable}
\end{longtable}


\begin{thebibliography}{2}

\bibitem{betaeta} Patrick S. Hagan, Diana E. Woodward, Markov interest rate models, Applied Mathematical Finance 6, 233–260 (1999)

\bibitem{piterbarg}Leif B. G. Andersen, Vladimir V. Piterbarg, Interest Rate Modeling, Atlantic Financial Press, 2010

\bibitem{ql}QuantLib A free/open-source library for quantitative finance, http://www.quantlib.org

\end{thebibliography}

\end{document}

