%% Based on a TeXnicCenter-Template by Gyorgy SZEIDL.
%%%%%%%%%%%%%%%%%%%%%%%%%%%%%%%%%%%%%%%%%%%%%%%%%%%%%%%%%%%%%

%------------------------------------------------------------
%
\documentclass{amsart}
%
%----------------------------------------------------------
% This is a sample document for the AMS LaTeX Article Class
% Class options
%        -- Point size:  8pt, 9pt, 10pt (default), 11pt, 12pt
%        -- Paper size:  letterpaper(default), a4paper
%        -- Orientation: portrait(default), landscape
%        -- Print size:  oneside, twoside(default)
%        -- Quality:     final(default), draft
%        -- Title page:  notitlepage, titlepage(default)
%        -- Start chapter on left:
%                        openright(default), openany
%        -- Columns:     onecolumn(default), twocolumn
%        -- Omit extra math features:
%                        nomath
%        -- AMSfonts:    noamsfonts
%        -- PSAMSFonts  (fewer AMSfonts sizes):
%                        psamsfonts
%        -- Equation numbering:
%                        leqno(default), reqno (equation numbers are on the right side)
%        -- Equation centering:
%                        centertags(default), tbtags
%        -- Displayed equations (centered is the default):
%                        fleqn (equations start at the same distance from the right side)
%        -- Electronic journal:
%                        e-only
%------------------------------------------------------------
% For instance the command
%          \documentclass[a4paper,12pt,reqno]{amsart}
% ensures that the paper size is a4, fonts are typeset at the size 12p
% and the equation numbers are on the right side
%
\usepackage{amsmath}%
\usepackage{amsfonts}%
\usepackage{amssymb}%
\usepackage{graphicx}
%------------------------------------------------------------
% Theorem like environments
%
\newtheorem{theorem}{Theorem}
\theoremstyle{plain}
\newtheorem{acknowledgement}{Acknowledgement}
\newtheorem{algorithm}{Algorithm}
\newtheorem{axiom}{Axiom}
\newtheorem{case}{Case}
\newtheorem{claim}{Claim}
\newtheorem{conclusion}{Conclusion}
\newtheorem{condition}{Condition}
\newtheorem{conjecture}{Conjecture}
\newtheorem{corollary}{Corollary}
\newtheorem{criterion}{Criterion}
\newtheorem{definition}{Definition}
\newtheorem{example}{Example}
\newtheorem{exercise}{Exercise}
\newtheorem{lemma}{Lemma}
\newtheorem{notation}{Notation}
\newtheorem{problem}{Problem}
\newtheorem{proposition}{Proposition}
\newtheorem{remark}{Remark}
\newtheorem{solution}{Solution}
\newtheorem{summary}{Summary}
\numberwithin{equation}{section}
%--------------------------------------------------------
\DeclareMathOperator{\sign}{sign}
%--------------------------------------------------------
\begin{document}
\title[$\beta-\eta$ Model Implementation]{$\beta-\eta$ Model Implementation}
\author{P. Caspers}
\author{R. Lichters}
\email[P. Caspers]{pcaspers1973@googlemail.com}
\email[R. Lichters]{pcaspers1973@googlemail.com}
\date{May 31, 2013}
\dedicatory{First Version May 31, 2015 - This Version June 16, 2015}
\begin{abstract}
We describe the implementation of the $\beta-\eta$ model \cite{betaeta}, \cite{piterbarg}, 11.3.2.6 in QuantLib \cite{ql}.
\end{abstract}

\maketitle

\section{Model}

The driving process is given by

\begin{equation}\label{dynamics}
dx(t) = \alpha(t) ( 1 + \beta x(t) )^\eta dW(t) 
\end{equation}

with $x(0)=0$, $\alpha(\cdot) > 0$, $\beta > 0$ and $0 \leq \eta \leq 1$. The dynamics is expressed in the measure $\mathbb{Q}^N$ associated to the numeraire

\begin{equation}
N(t,x(t)) = e^{\lambda(t)x(t)+M(0,0;T)}
\end{equation}

with a function $\lambda: \mathbb{R}\rightarrow\mathbb{R}$ subject to constraints $\lambda(0)=0$ and $\lambda'(0)=1$. We define

\begin{equation}\label{formula_M}
M(t,x;T) = \log E \left( e^{-\lambda(T)(X(T)-x)} \middle | X(t)=x \right)
\end{equation}

The transition density is given in \cite{betaeta}, (4.6a), (4.6b) and (4.8). For $\eta < 0.5$ a reflecting barrier at $x=-1/\beta$ can be specified in \ref{dynamics}, the corresponding amendments in the density are also given in \cite{betaeta}, p. 241, last paragraph.

For the special case of $\eta=0$ a closed form representation for the density (C.3) and for $M(t,x;T)$ (C.5) is given, but only for the case of no barrier. For $\eta=0.5$ a closed form expression for $M(t,x;T)$ is given in (4.15). For $\eta=1$ we can use (4.10b) likewise for the density (however there is no closed form expression for $M(t,x;T)$ in this case).


\section{Parametrization of $\lambda$}

We follow \cite{piterbarg} and describe $\lambda$ by a constant mean reversion $\kappa \neq 0$ via

\begin{equation}
\lambda = \frac{1 - e^{-\kappa t}}{\kappa}
\end{equation}

The normalization constraints $\lambda(0)=0, \lambda'(0)=1$ are immediately verified. Furthermore

\begin{equation}
\kappa = -\frac{\lambda''(t)}{\lambda'(t)}
\end{equation}

Note that $\lambda > 0$ by construction.

\section{Computation of $p(t,x;\overline{t},\overline{x})$}




\section{Tabulation of $M(t,x,T)$}

For $\eta=0.5$ we can use a closed form expression for $M$, see above. For $\eta=1$ we can compute

\begin{equation}
M(t,x;T) = \int_{-\infty}^\infty e^{-\lambda(T)\beta^{-1} (\exp(\beta \overline{y}) - \exp(\beta y))} e^{-z^2} dz
\end{equation}

where $\overline{y} = z\sqrt{2v} + y - \beta v / 2$, $v=\overline{\tau}-\tau$. Here we use the general transformation from \cite{betaeta}, (4.4)

\begin{equation}
y = \begin{cases}
\frac{|1+\beta x|^{1-\eta}}{\beta(1-\eta)} & \eta\neq 1 \\
\log(1+\beta x)/\beta & \eta=1 \\
\end{cases}
\end{equation}

and write 

\begin{equation}
\tau = \int_0^{t} \alpha(s) ds
\end{equation}

Overlined variables like $\overline{y}$ and $\overline{\tau}$ denote the quantities w.r.t. $T$ (or $\overline{t}$) instead of $t$. In the following we look at the case $\eta < 1$ (and $\eta \neq 0.5$, although the computations work for this very case as well).

The transition density is originally expressed in variables $y, \overline{y}$ and $\tau, \overline{\tau}$ instead of $x, \overline{x}$ and $t, \overline{t}$. It is obvious from \cite{betaeta}, (4.6a), (4.6b), (4.7) and (4.8) that $p$ can be written as a function of $\overline{\tau}-\tau$. Furthermore for any $a > 0$ we have the homogenity relation

\begin{equation}\label{phom}
p\left(a^2(\overline{\tau}-\tau), ay, a\overline{y}\right) = a^{1/(\eta-1)} p\left(\overline{\tau}-\tau, y, \overline{y}\right)
\end{equation}

We have to compute

\begin{equation}
M(t,x;\overline{t}) = \log \int_{-\infty}^{\infty} p(t,x;\overline{t},\overline{x}) e^{-\lambda(\overline{t})(\overline{x}-x)}d\overline{x}
\end{equation}

The integral can be rewritten using variable transformation and \ref{phom} as

\begin{equation}
\int_{0}^{\infty} p^* ( \lambda^{2-2\eta} \beta^{2\eta} (1-\eta)^{2\eta} v, u^{1-\eta}(1-\eta)^{\eta-1}, \overline{u}^{1-\eta}(1-\eta)^{\eta-1} ) e^{-(\overline{u}-u)} d\overline{u}
\end{equation}

where $p^*= p\cdot [(1-\eta)\beta]^{-\eta/(\eta-1)}$ with $u = \lambda(\overline{t})\beta^{-1}|1+\beta x|$. Introducing $S:=\beta^2v/(1+\beta x)^{2-2\eta}$ this reads

\begin{equation}
\int_{0}^{\infty} p^* ( S (1-\eta)^{2\eta} u^{2-2\eta}, u^{1-\eta}(1-\eta)^{\eta-1}, \overline{u}^{1-\eta}(1-\eta)^{\eta-1} ) e^{-(\overline{u}-u)} d\overline{u}
\end{equation}

this gets

\begin{equation}
(1-\eta)^{\eta/(\eta-1)} u^{-1} \int_{0}^{\infty} p^* ( S, (1-\eta)^{-1}, (\overline{u}/u)^{1-\eta}(1-\eta)^{-1} ) e^{-(\overline{u}-u)} d\overline{u}
\end{equation}

so $M = M(S,u)$ can be written as 

\begin{eqnarray}
\frac{\eta}{\eta-1} \log (1-\eta) - \log(u) + \\
\log \int_{0}^{\infty} p^* ( S, (1-\eta)^{-1}, (\overline{u}/u)^{1-\eta}(1-\eta)^{-1} ) e^{-(\overline{u}-u)} d\overline{u}
\end{eqnarray}

which can be tabulated.

Alternatively, the integral can be expressed as

\begin{equation}
\int_{0}^{\infty} \overline{w}^{\eta/(1-\eta)}p^*(1,w,\overline{w})e^{-(1-\eta)v^{*1/(2-2\eta)}(\overline{w}^{1/(1-\eta)}-w^{1/(1-\eta)})} d\overline{w}
\end{equation}

with $w=u^{1-\eta}(1-\eta)^{\eta-1}/\sqrt{v^*}$ and $v^*= \lambda^{2-2\eta} \beta^{2\eta} (1-\eta)^{2\eta} v$.



\begin{thebibliography}{2}

\bibitem{betaeta} Patrick S. Hagan, Diana E. Woodward, Markov interest rate models, Applied Mathematical Finance 6, 233–260 (1999)

\bibitem{piterbarg}Leif B. G. Andersen, Vladimir V. Piterbarg, Interest Rate Modeling, Atlantic Financial Press, 2010

\bibitem{ql}QuantLib A free/open-source library for quantitative finance, http://www.quantlib.org

\end{thebibliography}

\end{document}

