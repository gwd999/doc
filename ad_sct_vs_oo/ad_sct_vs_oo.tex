%% Based on a TeXnicCenter-Template by Gyorgy SZEIDL.
%%%%%%%%%%%%%%%%%%%%%%%%%%%%%%%%%%%%%%%%%%%%%%%%%%%%%%%%%%%%%

%------------------------------------------------------------
%
\documentclass{amsart}
%
%----------------------------------------------------------
% This is a sample document for the AMS LaTeX Article Class
% Class options
%        -- Point size:  8pt, 9pt, 10pt (default), 11pt, 12pt
%        -- Paper size:  letterpaper(default), a4paper
%        -- Orientation: portrait(default), landscape
%        -- Print size:  oneside, twoside(default)
%        -- Quality:     final(default), draft
%        -- Title page:  notitlepage, titlepage(default)
%        -- Start chapter on left:
%                        openright(default), openany
%        -- Columns:     onecolumn(default), twocolumn
%        -- Omit extra math features:
%                        nomath
%        -- AMSfonts:    noamsfonts
%        -- PSAMSFonts  (fewer AMSfonts sizes):
%                        psamsfonts
%        -- Equation numbering:
%                        leqno(default), reqno (equation numbers are on the right side)
%        -- Equation centering:
%                        centertags(default), tbtags
%        -- Displayed equations (centered is the default):
%                        fleqn (equations start at the same distance from the right side)
%        -- Electronic journal:
%                        e-only
%------------------------------------------------------------
% For instance the command
%          \documentclass[a4paper,12pt,reqno]{amsart}
% ensures that the paper size is a4, fonts are typeset at the size 12p
% and the equation numbers are on the right side
%
\usepackage{amsmath}%
\usepackage{amsfonts}%
\usepackage{amssymb}%
\usepackage{graphicx}
\usepackage[miktex]{gnuplottex}
\ShellEscapetrue
\usepackage{epstopdf}
\usepackage{longtable}
\usepackage{minted}
\usepackage{url}
\definecolor{mintedBg}{rgb}{0.98,0.98,0.70}
%------------------------------------------------------------
% Theorem like environments
%
\newtheorem{theorem}{Theorem}
\theoremstyle{plain}
\newtheorem{acknowledgement}{Acknowledgement}
\newtheorem{algorithm}{Algorithm}
\newtheorem{axiom}{Axiom}
\newtheorem{case}{Case}
\newtheorem{claim}{Claim}
\newtheorem{conclusion}{Conclusion}
\newtheorem{condition}{Condition}
\newtheorem{conjecture}{Conjecture}
\newtheorem{corollary}{Corollary}
\newtheorem{criterion}{Criterion}
\newtheorem{definition}{Definition}
\newtheorem{example}{Example}
\newtheorem{exercise}{Exercise}
\newtheorem{lemma}{Lemma}
\newtheorem{notation}{Notation}
\newtheorem{problem}{Problem}
\newtheorem{proposition}{Proposition}
\newtheorem{remark}{Remark}
\newtheorem{solution}{Solution}
\newtheorem{summary}{Summary}
\numberwithin{equation}{section}
%--------------------------------------------------------
\DeclareMathOperator{\sign}{sign}
%--------------------------------------------------------
\begin{document}
\title[AD]{Automatic Differentiation Case Study}
\author{P. Caspers}
\email[P. Caspers]{pcaspers1973@googlemail.com}
\date{July 12, 2015}
\dedicatory{First Version July 14, 2015 - This Version July 14, 2015}
\begin{abstract}
This is a case study in computing adjoint derivatives with (a) operator overloading using CppAD \cite{CppAD} and ADOL-C \cite{ADOLC} and (b) source code transformation using OpenAD/F \cite{OpenAD}
\end{abstract}

\maketitle

\tableofcontents

\section{Sample Problem}

We consider a local volatility asset process 

\begin{eqnarray}
dS &=& \sigma(S,t) S dW \\
d\log(S) &=& -\frac{1}{2}\sigma(S,t)^2 dt + \sigma(S,t) dW
\end{eqnarray}

and the valuation of an american call option with maturity $T$ and strike $K$ in this model. The call price $c = c(t,X), X:=\log(S)$ obeys

\begin{equation}
dc = \left(c_t + \frac{1}{2} \sigma(S,t)^2 (c_{XX}-c_{X}) \right) dt + c_X \sigma(S,t) dW 
\end{equation}

from where we get the backward PDE

\begin{equation}
c_t + \frac{1}{2}\sigma(S,t)^2(c_{XX}-c_X) = 0
\end{equation}

with initial condition $c(T,X(T)) = \max\{ e^{X(T)}  - K, 0 \}$.

We aim to compute the price using a simple explicit Euler finite difference scheme and the sensitivity to $\sigma(S,t)$ (vega) using the forward and reverse modes of \ref{CppAD}, \ref{ADOLC} and \ref{OpenAD}. 


\section{Test setup}

We implement a solution to the problem using the high level objects of \ref{ql}, in particular the finite difference framework and an american option pricing engine. This is solely to generate a reference price against we can check the other ad-hoc implementations.

Next we implement a low level C++ solution from scratch without any external libraries. The test code allows for three modes which are

\begin{enumerate}
\item \verb+PLAIN+ compute the option price using the standard \verb+double+ type
\item \verb+CPPAD+ compute the option price and reverse (forward) mode sensitivities to $\sigma(S,t)$ using CppAD
\item \verb+ADOLC+ compute the option price and reverse mode sensitivities to $\sigma(S,t)$ using ADOL-C
\end{enumerate}

The last implementation is in Fortran90 which can be fed into the source code transformation engine of OpenAD/F to generate a differentiated version of the pricing code. The implementation is separated into a core which is used from three drivers

\begin{enumerate}
\item \verb+driverf90_plain.f90+ to compute the option price using the standard \verb+double precision+ type
\item \verb+driverf90_forward.f90+ to compute forward mode sensitivities to $\sigma(S,t)$
\item \verb+dirverf90.f90+ to compute reverse mode sensitivities to $\sigma(S,t)$
\end{enumerate}

The parameters for the problem are chosen to be $T=10$, $S(0)=100$ and $K=120$. The PDE uses $5000$ time steps and $101$ grid points to discretize $\log(S)$ in the interval $[1.07555172964,8.13478864234]$ (these values are copied from the used QuantLib engine which determines the bounds automatically). We use a uniform grid in time and spatial direction, while QuantLib uses a concentrating mesher around the strike for the asset grid, a minor difference which we ignore and which might explain the slight pricing differences we observe below.

We use a flat volatility $\sigma(S,t) = 0.2$ for the pricing, but interpret the volatility to be piecewise constant on a grid of $500$ buckets (i.e. each bucket covers $10$ time steps in the pricing scheme) and compute the sensitivities to each of these buckets. This is to get a realistic test setup with a lot of sensitivities to compute, which is the main use case of the reverse mode of AD tools. Although we compute the bucketed vega we report only the sum of these bucket vegas in the results below.

For the forward mode tests we compute only the total vega (i.e. only one forward sweep is executed where the dot product of the partial derivatives and $(1,\dots,1)$ is computed).

\section{Test enviroment}

Here we list the test enviroment's main characteristics:

\begin{enumerate}
\item CPU: Intel(R) Core(TM) i7-2760QM CPU @ 2.40GHz, single threaded,
\item OS: Ubuntu 14.04
\item Compiler: gcc (g++, gfortran) 4.9.2, optimization O3
\item AD tools: ADOL-C-2.5.2, CppAD git repository 02a00a, Open AD/F svn revision 247
\end{enumerate}

\section{Test results}

\begin{table}[ht]
\begin{tabular}{l | r | r}
Code & NPV & Total Vega \\ \hline
QuantLib (pricing reference) & 18.4072 & na \\
from scratch C++ / CppAD & 18.3746 & 126.0181 \\
from scrach C++ / ADOL-C & 18.3746 & 126.0181   \\
Fortran / OpenAD/F & 18.3746 & 126.0181 \\
\end{tabular}
\caption{Pricing results Quantlib, from scratch C++, Fortran}
\end{table}

\begin{table}[ht]
\caption{Computation times (loop over 50 pde solutions) in milliseconds}
\begin{tabular}{l | r | r | r | r}
Test & QuantLib & f.s. C++ / CppAD & f.s. C++ / ADOL-C & Fortran / OpenAD/F \\ \hline
Plain computation & 37.4 & \multicolumn{2}{c|}{7.0} &  7.1 \\ \hline
Forward & na & 575.54 & 528.3 & 15.6 \\
Reverse & na & 540.9 & 544.8 & 31.3
\end{tabular}
\label{computationTimes}
\end{table}

\begin{table}[ht]
\caption{Computation time OpenAD/F averaged over different loop sizes in milliseconds}
\begin{tabular}{r | r}
Loop size & OpenAD/F Reverse \\ \hline
1 & 216.0 \\
2 & 125.0 \\
3 & 86.7 \\
4 & 76.5 \\
5 & 66.6 \\
10 & 47.9 \\
50 & 33.2 \\
100 & 30.11 \\
500 & 28.7 \\
1000 & 28.4 \\
5000 & 28.3
\end{tabular}
\end{table}

\section{Runtime problems with OpenAD/F}

\begin{minted}{fortran}
       if(j==0.or.j==sizeS-1) then
          d2=0
          d1=0 ! we do not really want this ...
       else
          d2=(c(swap,j+1)-2.0d0*c(swap,j)+c(swap,j-1))/hq
          d1 = (c(swap,j+1)-c(swap,j-1))/(2.0d0*h)
       endif
       ! ... but the following code does not build with OpenAD:
       ! if(j==0) d1 = (c(swap,j+1)-c(swap,j))/h
       ! if(j==sizeS-1) d1 = (c(swap,j)-c(swap,j-1))/h
       ! if(j>0.and.j<sizeS-1) d1 = (c(swap,j+1)-c(swap,j-1))/(2.0d0*h)
\end{minted}

\begin{minted}
Program received signal SIGSEGV: Segmentation fault - invalid memory reference.

Backtrace for this error:
#0  0x7F5305DDD3B7
#1  0x7F5305DDD9CE
#2  0x7F53052DCD3F
#3  0x41072F in toy_pde_ at testf90.pre.xb.x2w.w2f.post.f90:792
#4  0x40F472 in driver at driverf90.f90:25
Segmentation fault (core dumped)
\end{minted}


\begin{minted}{fortran}
integer, parameter:: sizes = 101, sizet = 7000
\end{minted}

\begin{minted}
Program received signal SIGBUS: Access to an undefined portion of a memory object.

Backtrace for this error:
#0  0x7F3E7E79B3B7
#1  0x7F3E7E79B9CE
#2  0x7F3E7DC9AD3F
#3  0x410394 in toy_pde_ at testf90.pre.xb.x2w.w2f.post.f90:601
#4  0x40F472 in driver at driverf90.f90:25
Bus error (core dumped)
\end{minted}

\begin{thebibliography}{4}

\bibitem{CppAD}\url{https://projects.coin-or.org/CPPAD}

\bibitem{ADOLC}\url{https://projects.coin-or.org/ADOL-C}

\bibitem{OpenAD}\url{http://www.mcs.anl.gov/OpenAD}

\bibitem{ql}QuantLib A free/open-source library for quantitative finance, \url{http://www.quantlib.org}

\end{thebibliography}

\begin{appendix}

\section{Source Code}

In this section we list the full source code used in this paper. Since the OpenAD/F framework requires non-standard build steps we list the used make files as well.

\subsection{CppAD and ADOL-C C++ test code}

File \verb+testcpp.cpp+

\begin{minted}{c++}
// define to enable AD
//#define CPPAD

// define to enable ADOL-C
#define ADOLC

// define for plain computation
//#define PLAIN

#include <iostream>
#include <vector>
#include <cmath>

#ifdef CPPAD
#include <cppad/cppad.hpp>
using CppAD::AD;
typedef AD<double> dbl;
#endif

#ifdef ADOLC
#include <adolc/adolc.h>
typedef adouble dbl;
#endif

#ifdef PLAIN
typedef double dbl;
#endif

// problem data
const dbl S0 = std::log(100.0);
const double T = 10.0;
const dbl K = std::log(120.0);
const unsigned int n = 500; // sigma grid

// PDE parameters
const dbl Smin = 1.07555172964;
const dbl Smax = 8.13478864234;
const unsigned int sizeS = 2 * 50 + 1;
const unsigned int sizeT = 500 * 10;

// solution grid
dbl loc[sizeS + 1], c[2][sizeS], exerciseValue[sizeS];

int main() {

#ifdef PLAIN
    std::vector<dbl> implVol(n, 0.20);
#endif

#ifdef CPPAD
    std::vector<dbl> implVol(n, 0.20);
    CppAD::Independent(implVol);
#endif

#ifdef ADOLC
    int tag = 1, keep = 1;
    adouble *implVol;
    implVol = new adouble[n];
    trace_on(tag, keep);
    for (unsigned int i = 0; i < n; ++i)
        implVol[i] <<= 0.20;
#endif

    unsigned int swap = 0;

    // initial values
    const dbl h = (Smax - Smin) / static_cast<dbl>(sizeS - 1);
    const dbl hq = h * h;
    for (unsigned int j = 0; j < sizeS; ++j) {
        loc[j] = Smin + h * static_cast<double>(j);
        c[swap][j] = exerciseValue[j] =
            std::max<dbl>(exp(loc[j]) - exp(K), 0.0);
    }

    // PDE solver
    const double dt = T / static_cast<double>(sizeT);
    for (unsigned int i = 0; i < sizeT; ++i) {
        // rollback
        for (unsigned int j = 0; j < sizeS; ++j) {
            const dbl v = implVol[static_cast<int>(static_cast<double>(i) *
                                                   static_cast<double>(n) /
                                                   static_cast<double>(sizeT))];
            dbl d1, d2;
            if (j == 0 || j == sizeS - 1) {
                d2 = 0.0;
            } else {
                d2 = (c[swap][j + 1] - 2.0 * c[swap][j] + c[swap][j - 1]) / hq;
            }
            if (j == 0) {
                d1 = (c[swap][j + 1] - c[swap][j]) / h;
            } else {
                if (j == sizeS - 1) {
                    d1 = (c[swap][j] - c[swap][j - 1]) / h;
                } else {
                    d1 = (c[swap][j + 1] - c[swap][j - 1]) / (2.0 * h);
                }
            }
            // Euler
            c[1 - swap][j] = c[swap][j] + 0.5 * dt * v * v * (d2 - d1);
        }
        // update prices
        for (unsigned int j = 0; j < sizeS; ++j) {
            c[1 - swap][j] = std::max(c[1 - swap][j], exerciseValue[j]);
        }
        swap = 1 - swap;
    }

    // solution output
    std::clog.precision(12);
    std::clog << "c(0,0) = " << c[swap][(sizeS - 1) / 2] << std::endl;

#ifdef CPPAD
    std::vector<dbl> y(1);
    y[0] = c[swap][(sizeS - 1) / 2];
    CppAD::ADFun<double> f(implVol, y);
    std::vector<double> w(1, 1.0);
    std::vector<double> vega(n);
    vega = f.Reverse(1, w);
    double sum = 0.0;
    for (unsigned int i = 0; i < n; ++i) {
        sum += vega[i];
    }
    // std::vector<double> x0(n, 1.0);
    // sum = f.Forward(1, x0)[0];
    std::clog << "vega =" << sum << std::endl;
#endif

#ifdef ADOLC
    double yout;
    c[swap][(sizeS - 1) / 2] >>= yout;
    trace_off();
    double u[1];
    u[0] = 1.0;
    double *vega = new double[n];
    reverse(tag, 1, n, 0, u, vega);
    double sum = 0.0;
    for (unsigned int i = 0; i < n; ++i) {
        sum += vega[i];
    }
    // double x0[n], x1[n], y0[1], y1[1];
    // for (unsigned int i = 0; i < n; ++i) {
    //    x0[i] = 0.2;
    //    x1[i] = 1.0;
    // }
    // fos_forward(tag, 1, n, 0, x0, x1, y0, y1);
    // sum = y1[0];
    std::clog << "vega =" << sum << std::endl;
    delete[] implVol;
    delete[] vega;
#endif

} // main
\end{minted}

\subsection{QuantLib C++ reference code}

File \verb+testql.cpp+:

\begin{minted}{c++}
#include <ql/quantlib.hpp>

#include <boost/make_shared.hpp>

using namespace QuantLib;

int main() {

    Date evalDate(13, July, 2015);
    Settings::instance().evaluationDate() = evalDate;
    Date maturity = evalDate + 10 * Years;

    const Real S0 = 100.0;
    const Real K = 120.0;
    const Real vol = 0.20;

    const Size tGrid = 500 * 10;
    const Size xGrid = 2 * 50 + 1;

    const Handle<Quote> S0_q(boost::make_shared<SimpleQuote>(S0));
    const Handle<BlackVolTermStructure> blackVol(
        boost::make_shared<BlackConstantVol>(evalDate, NullCalendar(), vol,
                                             Actual365Fixed()));
    Handle<YieldTermStructure> zeroyts(
        boost::make_shared<FlatForward>(evalDate, 0.0, Actual365Fixed()));

    boost::shared_ptr<StrikedTypePayoff> payoff =
        boost::make_shared<PlainVanillaPayoff>(Option::Call, K);
    boost::shared_ptr<Exercise> exercise =
        boost::make_shared<AmericanExercise>(maturity);
    boost::shared_ptr<VanillaOption> option =
        boost::make_shared<VanillaOption>(payoff, exercise);

    boost::shared_ptr<BlackScholesProcess> p =
        boost::make_shared<BlackScholesProcess>(S0_q, zeroyts, blackVol);

    boost::shared_ptr<FdBlackScholesVanillaEngine> engine =
        boost::make_shared<FdBlackScholesVanillaEngine>(
            p, tGrid, xGrid, 0, FdmSchemeDesc::ExplicitEuler());

    option->setPricingEngine(engine);

    std::clog << "c = " << option->NPV() << std::endl;
}
\end{minted}

\subsection{OpenAD/F calculation Fortran code}

File \verb+testf90.f90+:

\begin{minted}{fortran}
subroutine toy_pde(impliedVol,price)
implicit none

integer, parameter:: n = 500

double precision:: impliedVol(0:n-1), price

double precision, parameter:: s0=4.605170185988092d0, k=4.787491742782046, t=10.0d0
double precision, parameter:: smin=1.07555172964d0, smax=8.13478864234d0
integer, parameter:: sizes = 101, sizet = 5000

double precision:: loc(0:sizes), c(0:1,0:sizes-1), exerciseValue(0:sizes-1)
double precision:: h, hq, dt, d1, d2, v
integer:: swap, i, j, ind

!$openad INDEPENDENT(impliedVol)

swap = 0

! initial values
h = (smax-smin)/dble(sizes-1)
hq = h*h
do j=0,sizeS-1,1
    loc(j) = smin + h*dble(j)
    c(swap,j) = max(dexp(loc(j))-dexp(K),0.0d0)
    exerciseValue(j) = c(swap,j)
end do

! PDE solver
dt = t / dble(sizeT)
do i=0,sizeT-1,1
    do j=0,sizeS-1,1
       ind = int(dble(n)*dble(i)/dble(sizeT))
       v = impliedVol(ind)
       if(j==0.or.j==sizeS-1) then
          d2=0
          d1=0 ! we do not really want this ...
       else
          d2=(c(swap,j+1)-2.0d0*c(swap,j)+c(swap,j-1))/hq
          d1 = (c(swap,j+1)-c(swap,j-1))/(2.0d0*h)
       endif
       ! ... but the following code does not build with OpenAD:
       ! if(j==0) d1 = (c(swap,j+1)-c(swap,j))/h
       ! if(j==sizeS-1) d1 = (c(swap,j)-c(swap,j-1))/h
       ! if(j>0.and.j<sizeS-1) d1 = (c(swap,j+1)-c(swap,j-1))/(2.0d0*h)
       c(1-swap,j)=c(swap,j) + 0.5 * dt * v*v * (d2-d1)
     end do
     do j=0,sizeS-1,1
        c(1-swap,j)=max(c(1-swap,j),exerciseValue(j))
     end do
     swap = 1 -swap
end do
  
! return solution
price = c(swap,(sizeS-1)/2)

!$openad DEPENDENT(price)

end subroutine toy_pde
\end{minted}

\subsection{OpenAD/F plain mode driver Fortran code}

File \verb+driverf90_plain.f90+:

\begin{minted}{fortran}
program driver
implicit none
external toy_pde

integer,parameter::n=500
integer::i
double precision::impliedVol(0:n-1),price

do i=0,n-1,1
   impliedVol(i) = 0.20d0
end do

call toy_pde(impliedVol, price)

write(*,*) 'c = ', price

end program driver
\end{minted}

\subsection{OpenAD/F forward mode driver Fortran code}

File \verb+driverf90_forward.f90+:

\begin{minted}{fortran}
program driver
use OAD_active
implicit none
external toy_pde

integer, parameter:: n = 500

type(active)::impliedVol(0:n-1),price

double precision::vega
integer::i

do i=0,n-1,1
   impliedVol(i)%v=0.20d0
   impliedVol(i)%d=1.0
end do

call toy_pde(impliedVol, price)

vega = price%d

write(*,*) 'c = ', price%v, ' dprice/dvol = ', vega

end program driver
\end{minted}

\subsection{OpenAD/F reverse mode driver Fortran code}

File \verb+driverf90.f90+:

\begin{minted}{fortran}
program driver
use OAD_active
use OAD_rev
implicit none
external toy_pde

integer, parameter:: n = 500

type(active)::impliedVol(0:n-1),price

double precision::vega
integer::i

do i=0,n-1,1
   impliedVol(i)%v=0.20d0
end do

price%d=1.0d0
our_rev_mode%tape=.true.

call toy_pde(impliedVol, price)

do i=0,n-1,1
   vega = vega + impliedVol(i)%d
end do

write(*,*) 'c = ', price%v, ' dprice/dvol = ', vega

end program driver
\end{minted}

\subsection{OpenAD/F makefile plain mode}

File \verb+Makefile_plain+:

\begin{minted}{makefile}
ifndef F90C
F90C=gfortran
endif
driverf90_plain: driverf90_plain.o testf90.o
	${F90C} -O3 -o $@ $^
testf90.f90, driverf90_plain.f90:
%.o : %.f90
	${F90C} -g -O3 -o $@ -c $<
clean:
	rm -f testf90.o driverf90_plain.o
.PHONY: clean toolChain
\end{minted}

\subsection{OpenAD/F makefile forward mode}

File \verb+Makefile_forward+:

\begin{minted}{makefile}
ifndef F90C
F90C=gfortran
endif
RTSUPP=w2f__types OAD_active
driverf90_forward: $(addsuffix .o, $(RTSUPP)) driverf90_forward.o
                                              testf90.pre.xb.x2w.w2f.post.o
	${F90C} -O3 -o $@ $^
testf90.pre.xb.x2w.w2f.post.f90 $(addsuffix .f90, $(RTSUPP)) : toolChain
toolChain : testf90.f90
	openad -c -m f $<
%.o : %.f90
	${F90C} -g -O3 -o $@ -c $<
clean:
	rm -f ad_template* OAD_* w2f__* iaddr*
	rm -f testf90.pre* *.B *.xaif *.o *.mod driver driverE *~
.PHONY: clean toolChain
\end{minted}

\subsection{OpenAD/F makefile reverse mode}

File \verb+Makefile+:

\begin{minted}{makefile}
ifndef F90C
F90C=gfortran
endif
ifndef CC
CC=gcc
endif
RTSUPP=w2f__types OAD_active OAD_cp OAD_tape OAD_rev
driverf90: $(addsuffix .o, $(RTSUPP)) driverf90.o testf90.pre.xb.x2w.w2f.post.o
	${F90C} -O3 -o $@ $^
testf90.pre.xb.x2w.w2f.post.f90 $(addsuffix .f90, $(RTSUPP)) iaddr.c : toolChain
toolChain : testf90.f90
	openad -c -m rj $<
%.o : %.f90
	${F90C} -g -O3 -o $@ -c $<
%.o : %.c
	${CC} -g -O3 -o $@ -c $<
clean:
	rm -f ad_template* OAD_* w2f__* iaddr*
	rm -f testf90.pre* *.B *.xaif *.o *.mod driver driverE *~
.PHONY: clean toolChain
\end{minted}

\end{appendix}


\end{document}

