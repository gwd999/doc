\message{ !name(CallProbabilities.tex)}%% Based on a TeXnicCenter-Template by Gyorgy SZEIDL.
%%%%%%%%%%%%%%%%%%%%%%%%%%%%%%%%%%%%%%%%%%%%%%%%%%%%%%%%%%%%%

%------------------------------------------------------------
%
\documentclass{amsart}
%
%----------------------------------------------------------
% This is a sample document for the AMS LaTeX Article Class
% Class options
%        -- Point size:  8pt, 9pt, 10pt (default), 11pt, 12pt
%        -- Paper size:  letterpaper(default), a4paper
%        -- Orientation: portrait(default), landscape
%        -- Print size:  oneside, twoside(default)
%        -- Quality:     final(default), draft
%        -- Title page:  notitlepage, titlepage(default)
%        -- Start chapter on left:
%                        openright(default), openany
%        -- Columns:     onecolumn(default), twocolumn
%        -- Omit extra math features:
%                        nomath
%        -- AMSfonts:    noamsfonts
%        -- PSAMSFonts  (fewer AMSfonts sizes):
%                        psamsfonts
%        -- Equation numbering:
%                        leqno(default), reqno (equation numbers are on the right side)
%        -- Equation centering:
%                        centertags(default), tbtags
%        -- Displayed equations (centered is the default):
%                        fleqn (equations start at the same distance from the right side)
%        -- Electronic journal:
%                        e-only
%------------------------------------------------------------
% For instance the command
%          \documentclass[a4paper,12pt,reqno]{amsart}
% ensures that the paper size is a4, fonts are typeset at the size 12p
% and the equation numbers are on the right side
%
\usepackage{amsmath}%
\usepackage{amsfonts}%
\usepackage{amssymb}%
\usepackage{graphicx}
\usepackage[miktex]{gnuplottex}
\usepackage{epstopdf}
%\usepackage[pdf]{pstricks}
%------------------------------------------------------------
% Theorem like environments
%
\newtheorem{theorem}{Theorem}
\theoremstyle{plain}
\newtheorem{acknowledgement}{Acknowledgement}
\newtheorem{algorithm}{Algorithm}
\newtheorem{axiom}{Axiom}
\newtheorem{case}{Case}
\newtheorem{claim}{Claim}
\newtheorem{conclusion}{Conclusion}
\newtheorem{condition}{Condition}
\newtheorem{conjecture}{Conjecture}
\newtheorem{corollary}{Corollary}
\newtheorem{criterion}{Criterion}
\newtheorem{definition}{Definition}
\newtheorem{example}{Example}
\newtheorem{exercise}{Exercise}
\newtheorem{lemma}{Lemma}
\newtheorem{notation}{Notation}
\newtheorem{problem}{Problem}
\newtheorem{proposition}{Proposition}
\newtheorem{remark}{Remark}
\newtheorem{solution}{Solution}
\newtheorem{summary}{Summary}
\numberwithin{equation}{section}
%--------------------------------------------------------

\begin{document}

\message{ !name(CallProbabilities.tex) !offset(-3) }


\title[Call Probabilities]{Extracting call probabilities from pricing models}
\author{Peter Caspers}
\email[Peter Caspers]{pcaspers1973@gmail.com}
\date{May 1, 2013}
\dedicatory{First Version May 1, 2013 - This Version May 1, 2013}
\begin{abstract}
We demonstrate the strong dependency of call probabilities on the pricing measure taking the example
of an european swaption in a Hull White one factor model under different $T$ forward measures. As a remedy we suggest a definition
that makes these probabilities well defined and unique.
\end{abstract}
\maketitle

\section{Introduction}

It is common practice to extract exercise probabilities from pricing models directly under the repective pricing measure. For pricing purposes
very different measures may be used however, such as the risk neutral measure, the $T$-forward measure,
the spot measure, the swap annuity measure. It is known from the general theory that these measures are equivalent
meaning that they share the same sets with probability zero (or one). The probabilities of all events not sure or impossible
however may differ significantly.

This paper is meant to demonstrate this on the example of the exercise probability of an european swaption in a Hull
White one factor model under different $T$-forward measures. The lesson learnt here is that probabilities should not extracted from a pricing model directly,
because they are measure dependent thus not well defined.

We instead suggest to define exercise probabilities in terms of forward pricies of certain digital options. This definition
coincides with the naive probabilities in the $T$-forward measure with $T$ being the exercise time.

\section{Hull White forward dynamics}

The dynamics of the Hull White one factor model with short rate $r$ under the $T$-forward measure may be written as 

\begin{equation}\label{hwdyn}
dx(t) = (y(t)-\sigma_r(t)^2G(t,T)-\kappa(t)x(t))dt+\sigma_r(t)dW^T(t)
\end{equation}

with $x(t) := r(t) - f(0,t)$ and $\sigma$ and $\kappa$ denoting the model volatility and reversion speed and

\begin{eqnarray}
y(t) = \int_0^t e^{-2 \int_u^t \kappa(s) ds} \sigma_r(u)^2 du \\
G(t,t') = \int_t^{t'} e^{-\int_t^u\kappa(s)ds} du
\end{eqnarray}

see e.g. \cite{piterbarg} for more details. The general pricing formula for a payoff $\Pi$ occuring at time $t$ in this setting reads

\begin{equation}
\nu = P(0,T) E^T \left( \frac{\Pi(t)}{P(t,T)} \right)
\end{equation}

The price $\nu$ itself does not depend on the choice of $T$, but the random variable inside the expectation does and hence 
in particular the (naive) exercise probability $P(\Pi(t) > 0 )$ (in case $\Pi$ represents a long option). In the following we display our example results in terms of the normalized state
variable

\begin{equation}
z(t) := \frac{x(t) - E(x(t))}{\sqrt{V(x(t))}}
\end{equation}

with $E, V$ denoting the expecation and variance operators. $z$ is by construction a standard normal distributed variable.

\section{Well defined exercise probabilities}

Instead of using naive, measure dependent exercise probabilities we suggest to use certain forward digital prices. To do so we define (for an arbitrary numeraire $N$)

\begin{equation}\label{defpi}
\pi := \frac{N(0)}{P(0,t)} E^N \left( \frac{1_{\Pi(t)>0}}{N(t)} \right)
\end{equation}

and interpret $\pi$ to be an exercise probability at time $t$ (where exercise as above is triggerd if and only if $\Pi(t)>0$). 

The naive exercise probability in contrast is defined as

\begin{equation}\label{defp}
p = p_N := E^N(1_{\Pi(t)>0})
\end{equation}

and dependent on the measure associated to $N$ as indicated by the subscript.

Since $\pi$ is a price it is well defined, i.e. measure independent. Furthermore $\pi$ conincides with the naive exercise probability for the special case of $N(\cdot)=P(\cdot,t)$, i.e.

\begin{equation}
\pi = E^t ( 1_{\Pi(t)>0} )
\end{equation}

where $t$ is the exercise time.

\section{Numerical examples}

We consider the valuation date 30-04-2013 and price an european at the money payer swaption with expiry in $5$ years and underlying swap of $10$ years.
The implied volatility is $20\%$ and the yield curve is flat forward at $2\%$ (continously compounded rate). The mean reversion in the Hull White
model is set to $1\%$ and the volatility is calibrated to $0.00422515$ in order to reprice the market swaption.

We compute the vanilla price $\nu$, the (naive) exercise probability $p$ within the pricing measure as defined in \ref{defp} and the forward digital price $\pi$ as defined in \ref{defpi}.

The computations are done using the libraries \cite{ql} and \cite{ntl}. Prices are computed using numerical integration covering $10$ standard deviations of the normal density.

Figure \ref{payoff} shows the functional dependency of the (undeflated) underlying payoff to the normalized state variable $z$ as defined above. The option is exercised if and only if the payoff is positive. With $T$ growing the crossing point of the payoff with the $z$ - axis shifts to the right. This is due to the $T$ dependent drift term $\sigma_r^2 G(\cdot,T)$ in \ref{hwdyn} and is in direct correspondence to the observation that the naive exercise probability is decreasing for increasing $T$. For pricing this is compensated by the factor $P(0,T)/P(t,T)$ as can be seen in figure \ref{payofffactor}.

\begin{table}[ht]
\caption{Numerical Example with vanilla price $\nu$, naive exercise probability $p$ and forward digital $\pi$}
\begin{tabular}{l | l | l | l}
$T$ & $\nu$ & $p$ & $\pi$ \\ \hline
16 & 0.0290412 & 0.478684 & 0.481185 \\
30 & 0.0290385 & 0.435909 & 0.484888 \\
60 & 0.0290381 & 0.364131 & 0.480800 \\
100 & 0.0290412 & 0.30056 & 0.480948 \\
250 & 0.0290415 & 0.211921 & 0.479532
\end{tabular}
\label{numex}
\end{table}

\begin{figure}[htbp]
\caption{Undeflated underlying payoff within $3$ standard deviations for $T=16$ (red) and $T=250$ (green)}
\label{payoff}
	\begin{gnuplot}
		set terminal epslatex color
		set xrange [-3:3]
		set yrange [-0.5:0.5]
		unset key
		set grid
		set xlabel "z"
		set ylabel "undeflated payoff"
		plot 'example.txt' using 1:2 with line, 'example.txt' using 1:6 with line
	\end{gnuplot}
\end{figure}

\begin{figure}[htbp]
\caption{$P(0,T)/P(t,T)$ within $3$ standard deviations for $T=16$ (red) and $T=250$ (green)}
\label{payofffactor}
	\begin{gnuplot}
		set terminal epslatex color
		set xrange [-3:3]
		set yrange [-0:10]
		unset key
		set grid
		set xlabel "z"
		set ylabel "P(0,T)/P(t,T)"
		plot 'example2.txt' using 1:2 with line, 'example2.txt' using 1:6 with line
	\end{gnuplot}
\end{figure}

\begin{thebibliography}{2}

\bibitem{ql}QuantLib A free/open-source library for quantitative finance, http://www.quantlib.org

\bibitem{ntl}Shoup, Victor: NTL A Library for doing Number theory, http://www.shoup.net/ntl/

\bibitem{piterbarg} Piterbarg, Andersen: Interest Rate Modeling, Volume I, Atlantic Financial Press, 2010


\end{thebibliography}

\end{document}



\message{ !name(CallProbabilities.tex) !offset(-233) }
