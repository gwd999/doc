\documentclass{beamer}

\usetheme{CambridgeUS}
\usefonttheme{professionalfonts}

\usepackage{graphicx}
\usepackage[miktex]{gnuplottex}
\ShellEscapetrue
\usepackage{epstopdf}
\usepackage{minted}

\usemintedstyle{manni}
\definecolor{mintedBg}{rgb}{0.98,0.98,0.70}

\begin{document}
\title{Negative rates in QuantLib}
\author{Peter Caspers}
\institute{Quaternion}
\date{November 30, 2015}

\frame{\titlepage}

\frame{\frametitle{Table of contents}\tableofcontents[hideallsubsections]}

\section{Negative fixings and implications}

\begin{frame}[fragile]
\frametitle{History of negative fixings}
\begin{itemize}
\item it started with negative EONIA fixings end of 2014
\item then we had negative Euribor 1m, later 3m, even 6m fixings
\item as of 27-Oct-2015 we have a negative CMS2Y fixing (at -3.5 bp)
\end{itemize}
\end{frame}

\begin{frame}[fragile]
\frametitle{Implications of negative fixings}
\begin{itemize}
\item interest compounding on collateral accounts, ISDA negative rates protocol
\item payment reversal in swaps under ISDA and DRV
\item floored coupons for bonds, schuldscheindarlehen, loans, ...
\end{itemize}
\end{frame}

\begin{frame}[fragile]
\frametitle{Implications on pricing}
\begin{itemize}
\item rate curves should allow for negative forwards
\item lognormal models can not reproduce market prices for zero (or negative strike) floors
\item lognormal models can even fail to produce high enough prices for boring forward levels like $F=1\%$ or $2\%$, because e.g. for shifted lognormal models with shift $d\geq0$, $c(K)/N(0)\rightarrow F+d$ if $\sigma\rightarrow\infty$. You could actually observe this recently by first exploding, then missing implied lognormal volatility quotes for EUR swaptions with long option tenor
\end{itemize}
\end{frame}

\begin{frame}[fragile]
\frametitle{Implications on pricing}
\begin{itemize}
\item shifted Black76 and normal Black76 models were established as market models for low and negative rates
\item shifting is generic, e.g. the shifted SABR model has also become part of the new basic standard of market models
\item with a different motivation (produce skew) a shift was introduced in Libor forward models a long time ago
\item new models / model variants are discovered to handle negative rates in a more sophisticated way (free boundary SABR, mixed SABR)
\item other models need adjustments as well (cms replication coupon pricers, Markov functional model)
\end{itemize}
\end{frame}

\section{QuantLib implementation}

\begin{frame}[fragile]
\frametitle{Negative rates switch}
\begin{itemize}
\item \verb+QL_NEGATIVE_RATES+
\item allows for negative zero yields, forwards, increasing discount factors
\end{itemize}
\begin{minted}[bgcolor=mintedBg,fontsize=\footnotesize]{diff}
+2012-07-31 14:11  Ferdinando Ametrano
+
+	* [r18305] ql/userconfig.hpp, test-suite/piecewiseyieldcurve.cpp:
+
+     defaulted to allow negative rates (define QL_NEGATIVE_RATES) as this
+     is happening for EUR OIS, CHF and German treasury yields, etc.
\end{minted}
\end{frame}


\begin{frame}[fragile]
\frametitle{Volatility type}
\begin{itemize}
\item \verb+ql/termstructures/volatility/volatilitytype.hpp+
\item distinguishes between normal and (shifted) lognormal volatilities
\end{itemize}
\begin{minted}[bgcolor=mintedBg,fontsize=\footnotesize]{c++}
    enum VolatilityType { ShiftedLognormal, Normal };
\end{minted}
\end{frame}

\begin{frame}[fragile]
\frametitle{Cap Floor Volatilities}
\begin{itemize}
\item market quotes normal or shifted lognormal volatilities, with a constant shift across strikes and tenors
\end{itemize}
\begin{minted}[bgcolor=mintedBg,fontsize=\footnotesize]{c++}
OptionletStripper(const boost::shared_ptr<CapFloorTermVolSurface>&,
                  const boost::shared_ptr<IborIndex>& iborIndex_,
                  const Handle<YieldTermStructure>& discount =
                                         Handle<YieldTermStructure>(),
                  const VolatilityType type = ShiftedLognormal,
                  const Real displacement = 0.0);
\end{minted}
\end{frame}

\begin{frame}[fragile]
\frametitle{Swaption Volatilities}
\begin{itemize}
\item market quotes normal or shifted lognormal volatilities, with different shifts per underlying
\item swaption cubes inherit the shift structure from their embedded atm matrix
\item swaption volatility cube 1 uses shifted SABR models
\item the shift is bilinearly interpolated in (option, underlying) space
\end{itemize}
\begin{minted}[bgcolor=mintedBg,fontsize=\footnotesize]{c++}
SwaptionVolatilityMatrix(
            const Calendar& calendar,
            BusinessDayConvention bdc,
            ...
            const VolatilityType type = ShiftedLognormal,
            const std::vector<std::vector<Real> >& shifts
                        = std::vector<std::vector<Real> >());
\end{minted}
\end{frame}

\begin{frame}[fragile]
\frametitle{Libor in arrears adjustments}
\begin{itemize}
\item convexity adjustment is amended in a straightforward way for  shifted lognormal or normal volatilities
\item timing adjustment is generalized at the same time for arbitrary non-natural fixing times\footnote{see \url{http://ssrn.com/abstract=2170721}}
\end{itemize}
\begin{minted}[bgcolor=mintedBg,fontsize=\footnotesize]{c++}
enum TimingAdjustment { Black76,
                        BivariateLognormal };
BlackIborCouponPricer(const Handle<OptionletVolatilityStructure> &v =
                   Handle<OptionletVolatilityStructure>(),
                   const TimingAdjustment timingAdjustment = Black76,
                   const Handle<Quote> correlation =
                   Handle<Quote>(boost::make_shared<SimpleQuote>(1.0)))
\end{minted}
\end{frame}

\begin{frame}[fragile]
\frametitle{Linear TSR pricer}
\begin{itemize}
\item volatility type is recognized through the abstraction of \verb+SmileSection+
\item the replication range is shifted by the appropriate (i.e. user bounds set to $[0,200\%]$ and transformed to $[-1\%,199\%]$ automatically if the applicable shift is $1\%$ (to keep the user input universal under changing shifts in market quotations)
\end{itemize}
\end{frame}

\begin{frame}[fragile]
\frametitle{CMS spread option pricer}
\begin{itemize}
\item swap rate adjustments use shifted lognormal or normal smiles to determine the drifts of the single swap rate models
\item the bivariate model for the swap rates is still purely lognormal currently
\item with negative 2Y fixings, we will neeed to extend this pricer as well
\item plan: allow for shifts in the single rate models or for normal single rate models
\end{itemize}
\end{frame}

\begin{frame}[fragile]
\frametitle{Calibration helpers}
\begin{itemize}
\item can be set up with normal and shifted lognormal volatilities
\item cooperative with \verb+HullWhite+, \verb+Gsr+, \verb+Lgm+, \verb+MarkovFunctional+ models
\end{itemize}
\begin{minted}[bgcolor=mintedBg,fontsize=\footnotesize]{c++}
SwaptionHelper(const Period& maturity,
               const Period& length,
               const Handle<Quote>& volatility,
               ...
               const VolatilityType type = ShiftedLognormal,
               const Real shift = 0.0);
\end{minted}
\end{frame}

\begin{frame}[fragile]
\frametitle{Markov functional model}
\begin{itemize}
\item replicates a market smile / density per expiry via the numeraire calibration
\item therefore also replicates the density for negative strike ranges
\item currently, only shifted lognormal smile input allowed
\item todo: allow normal smile input for numeraire calibration
\end{itemize}
\end{frame}

\section{Questions}

\begin{frame}[fragile]
\frametitle{Questions / Discussion}
\centerline{thank you for your attention}
\end{frame}
\end{document}
