%% Based on a TeXnicCenter-Template by Gyorgy SZEIDL.
%%%%%%%%%%%%%%%%%%%%%%%%%%%%%%%%%%%%%%%%%%%%%%%%%%%%%%%%%%%%%

%------------------------------------------------------------
%
\documentclass{amsart}
%
%----------------------------------------------------------
% This is a sample document for the AMS LaTeX Article Class
% Class options
%        -- Point size:  8pt, 9pt, 10pt (default), 11pt, 12pt
%        -- Paper size:  letterpaper(default), a4paper
%        -- Orientation: portrait(default), landscape
%        -- Print size:  oneside, twoside(default)
%        -- Quality:     final(default), draft
%        -- Title page:  notitlepage, titlepage(default)
%        -- Start chapter on left:
%                        openright(default), openany
%        -- Columns:     onecolumn(default), twocolumn
%        -- Omit extra math features:
%                        nomath
%        -- AMSfonts:    noamsfonts
%        -- PSAMSFonts  (fewer AMSfonts sizes):
%                        psamsfonts
%        -- Equation numbering:
%                        leqno(default), reqno (equation numbers are on the right side)
%        -- Equation centering:
%                        centertags(default), tbtags
%        -- Displayed equations (centered is the default):
%                        fleqn (equations start at the same distance from the right side)
%        -- Electronic journal:
%                        e-only
%------------------------------------------------------------
% For instance the command
%          \documentclass[a4paper,12pt,reqno]{amsart}
% ensures that the paper size is a4, fonts are typeset at the size 12p
% and the equation numbers are on the right side
%
\usepackage{amsmath}%
\usepackage{amsfonts}%
\usepackage{amssymb}%
\usepackage{graphicx}
\usepackage{gnuplottex}
\usepackage{epstopdf}
%\usepackage[pdf]{pstricks}
%------------------------------------------------------------
% Theorem like environments
%
\newtheorem{theorem}{Theorem}
\theoremstyle{plain}
\newtheorem{acknowledgement}{Acknowledgement}
\newtheorem{algorithm}{Algorithm}
\newtheorem{axiom}{Axiom}
\newtheorem{case}{Case}
\newtheorem{claim}{Claim}
\newtheorem{conclusion}{Conclusion}
\newtheorem{condition}{Condition}
\newtheorem{conjecture}{Conjecture}
\newtheorem{corollary}{Corollary}
\newtheorem{criterion}{Criterion}
\newtheorem{definition}{Definition}
\newtheorem{example}{Example}
\newtheorem{exercise}{Exercise}
\newtheorem{lemma}{Lemma}
\newtheorem{notation}{Notation}
\newtheorem{problem}{Problem}
\newtheorem{proposition}{Proposition}
\newtheorem{remark}{Remark}
\newtheorem{solution}{Solution}
\newtheorem{summary}{Summary}
\numberwithin{equation}{section}
%-----------------------------------------------------------------
\begin{document}

\title{Implementation of the ZABR model}
\author{Peter Caspers}
\date{September 7, 2013}
\dedicatory{First Version July 17, 2012 - This Version September 7, 2013}
\maketitle

\begin{abstract}
We describe the implementation of the ZABR model \cite{zabr} and give numerical examples.
\end{abstract}

\section{Model description}
We follow \cite{zabr}. The general model description is given by

\begin{eqnarray}\label{generalmodeldef}
df(t) &=& \sigma(f(t)) v(t) dW(t) \\
dv(t) &=& \epsilon(v(t)) dV(t) \\
dW(t) dV(t) &=& \rho dt \\
v(0) &=& \alpha > 0
\end{eqnarray}

where the local volatility function $\sigma$ only depends on the forward $f$ (not on the time $t$) and is multiplied by a
stochastic volatility factor $v$, which is driven by a correlated brownian motion $V$, again with a space but not time dependent local volatility function $\epsilon$.

The SABR model is recoverd as a special case by setting

\begin{eqnarray}
\sigma(f) &=& f^{\beta} \\
\epsilon(v) &=& \eta v
\end{eqnarray}

with CEV parameter $\beta \in [0,1]$ and volatility of volatility $\eta \geq 0$.

\section{Implied normal volatility}

The aim is to derive a formula for european call option prices in the general model given by the general formula

\begin{equation}
c(t) = E( (f(T)-K)^+ | \mathcal{F}_t )
\end{equation}

where $T$ is the maturity, $K$ is the strike and $c(t)$ is the (non discounted) call price at time $t$.

We define the normal implied volatility $\nu(t)$ by

\begin{equation} \label{impliedvol}
B(f(t),\nu(t),\tau,K) = c(t)
\end{equation}

where $\tau := T-t$ is the time to maturity and $B$ is the normal Black Formula

\begin{equation}
B(f,\nu,\tau,K) = (f-K) \Phi \left( \frac{f-K}{\nu\sqrt{\tau}} \right) + 
                                   \nu\sqrt{\tau} \phi \left( \frac{f-K}{\nu\sqrt{\tau}} \right)
\end{equation}

The Ito formula applied to \ref{impliedvol} yields

\begin{equation} \label{cdynamic}
dc = -B_\tau dt + B_f df + B_\nu d\nu + \frac{1}{2} \left( B_{\nu\nu} d\nu d\nu + B_{ff} df df \right) + B_{f\nu} df d\nu 
\end{equation}

Setting 

\begin{equation}
x(t) := \frac{f(t)-K}{\nu(t)}
\end{equation}

the ito formula gives

\begin{equation}
dx = \frac{1}{\nu} df - \frac{f-K}{\nu^2} d\nu + \frac{f-K}{\nu^3} d\nu d\nu - \frac{1}{\nu^2} dfd\nu
 = \frac{1}{\nu} ( df - x d\nu ) + \textnormal{... }dt
\end{equation} 

because $dZ dt = dt dt = 0$ for every brownian $Z$. Therefore the quadratic variation of $x$ evaluates to

\begin{equation} \label{quadVarX}
dx dx = \frac{1}{\nu^2} ( df df - 2 x df d\nu + x^2 d\nu d\nu )
\end{equation}

The following equations are verified by direct computation

\begin{eqnarray} \label{propB}
B_\nu &=& \nu \tau B_{ff} \\
B_{\nu\nu} &=& \left( \frac{f-K}{\nu} \right) ^2 B_{ff} \\
B_{f \nu} &=& - \left( \frac{f-K}{\nu} \right) B_{ff} \\
0 &=& -B_\tau + \frac{1}{2} \nu^2 B_{ff}
\end{eqnarray}

Since $c$ and $f$ are martingales, taking expectations in \ref{cdynamic} and using \ref{propB} yields

\begin{equation}
0 = -\frac{1}{2} \nu^2 B_{ff} dt + \nu \tau B_{ff} d\nu  + \frac{1}{2} \left( x^2 B_{ff} d\nu d\nu + B_{ff} df df \right)
       - x B_{ff} df d\nu
\end{equation}

Dividing by $B_{ff}$ and using \ref{quadVarX} gives

\begin{equation}
0 = -\frac{1}{2} \nu^2 dt + \nu \tau E(d\nu) + \frac{1}{2} \nu^2 dx dx 
\end{equation}

which for $\tau \rightarrow 0$ becomes

\begin{equation}
0 = - \nu^2 ( dt - dx dx )
\end{equation}

or

\begin{equation}
\sigma_x := \frac{dx dx}{dt} = 1
\end{equation}

The aim is to find $x(t)$, then the normal implied volatility is given by

\begin{equation}
\nu(t) = \frac{f-K}{x(t)} 
\end{equation}

Setting $x(t) = x(f(t),v(t))$, one gets $dx=x_fdf+x_vdv+\textnormal{... }dt$, therefore

\begin{equation}\label{pde}
1 = \sigma_x = \frac{dx dx}{dt} = x_f^2\sigma(f)^2v^2+x_v^2\epsilon(v)^2+2x_fx_v\rho\sigma(f) v \epsilon(v)
\end{equation}

This is a nonlinear partial differential equation in the dummy variables $(fv)$, which can be solved using boundary condition

\begin{equation}
x(f=K,v) = 0
\end{equation}

\section{Deterministic case}

If $\epsilon(v)=0$, i.e. there is no stochastic volatility, the PDE \ref{pde} becomes

\begin{equation}
1 = x_s^2\sigma(f)^2v^2
\end{equation}

We can set $v=1$, taking the root and get

\begin{equation}
x_f = \sigma(f)^{-1}
\end{equation}

with solution under the boundary condition $x(K) = 0$

\begin{equation}
x = \int_{K}^f \sigma(u)^{-1} du
\end{equation}

thereby yielding the normal volatility

\begin{equation}
\nu = \frac{f-K}{\int_{K}^f \sigma(u)^{-1} du}
\end{equation}

Note that in the case $f=K$, $\nu=\sigma(f)$. In the lognormal case

\begin{equation}
\nu = \frac{\ln \frac{f}{K}} {\int_{K}^f \sigma(u)^{-1} du}
\end{equation}

where we have to take $\nu = \sigma(f) f^{-1}$ for $f=K$. The local volatility function $\sigma(f)$ can obviously reconstructed from $x$ by

\begin{equation}\label{localvolfromx}
\sigma(f) = - \left( \frac{\partial x}{\partial K}(f) \right) ^{-1}
\end{equation}


\section{The SABR model}

The SABR model is recovered by setting $\sigma(f)=f^\beta$ and $\epsilon(v)=\eta v$ with $\beta \in [0,1]$ and a volatility of volatility parameter $\eta \geq 0$.
The pde \ref{pde} now takes the form

\begin{equation}\label{pdesabr}
1 = x_f^2 f^{2\beta} v^2 + x_v^2 \eta^2 v^2 + 2 x_f x_v \rho f^\beta \eta v
\end{equation}

We define a new variable $y$ by setting

\begin{equation}\label{defsabry}
y(t) := v(t)^{-1}\int_{K}^{f(t)} \sigma(u)^{-1} du
\end{equation}

Ito gives

\begin{eqnarray}
dy = y_f df + y_v dv + \textnormal{... } dt = \\
 dW - \eta y(t) dV + \textnormal{... } dt = \\
 \sqrt{ 1 + \eta^2 y(t)^2 - 2 \rho\eta y(t) }  dB + \textnormal{... } dt =: \\
 J(y(t)) dB + \textnormal{... } dt
\end{eqnarray}

It should be noted that

\begin{equation}
1+\eta^2 y(t)^2 - 2\rho\eta y(t) = (\eta y(t) - \rho)^2 + (\rho^2-1) > 0
\end{equation}

whenever $\rho \in (-1,1)$ what we assume in the following. Now set

\begin{equation}\label{defsabrx}
x(t):=\int_0^{y(t)} J(u)^{-1} du
\end{equation}

We aim to show that $x$ satisfies pde \ref{pdesabr}. We compute

\begin{eqnarray}
y_f(t) &=& v(t)^{-1} \sigma(f(t))^{-1} \\
y_v(t) &=& -v(t)^{-1} y(t) \\
x_f(t) &=& J(y(t))^{-1} v(t)^{-1} \sigma(f(t))^{-1} \\
x_v(t) &=& -J(y(t))^{-1} v(t)^{-1} y(t)
\end{eqnarray}

Putting this into the rhs of \ref{pdesabr} we get

\begin{eqnarray}
J(y(t))^{-2} + J(y(t))^{-2} v(t)^{-2} y(t)^2 \epsilon(v)^2 - \\
2 J(y(t))^{-2} v(t)^{-2} y(t) \rho v(t) \epsilon(v) = \\
J(y(t))^{-2} ( 1 + \eta^2 y(t)^2 - 2 \eta \rho y(t) ) = 1 
\end{eqnarray} 

which means that $x(t)$ is a solution of \ref{pdesabr}. We note that only the special form of
$\epsilon(v) = \eta v$ was used to arrive here, whereas $\sigma(f)$ could have been arbitrary.

We explicity compute $x(t)$ from its definition \ref{defsabrx}. Using the elementary identity

\begin{equation}
\int \frac{dh}{\sqrt{z^2+a}} = \ln \frac{\sqrt{z^2+a}+z}{\sqrt{a}}
\end{equation}

which holds for $a>0$ we get by a simple change of variable $\eta y(t) - \rho =: z$

\begin{equation}
x(t) = \frac{1}{\eta} \ln \frac{J(y(t))+\eta y(t) - \rho}{1-\rho}
\end{equation}

where $y(t)$ is directly computed from its definition as

\begin{equation}
y(t) = \frac{f(t)^{1-\beta}-K^{1-\beta}}{v(t)(1-\beta)}
\end{equation}

if $\beta < 1$ and

\begin{equation}
y(t) = \frac{1}{v(t)} \ln \frac{f(t)}{K}
\end{equation}

if $\beta=1$, where $v(0)=\alpha$. The implied normal volatility is then retrieved as above as

\begin{equation}
\nu(t) = \frac{f(t)-K}{x(t)}
\end{equation}

This formula only holds for $K \neq f(t)$. The at the money case can be retrieved by differentiating the nominator
and denominator by $K$ and evaluating the result at $K=f(t)$. The derivative of the nominator is clearly $-1$ (in the
lognormal case $-K^{-1}$). The denominator is handled as follows (subscripts denoting partial derivatives and omitting
independent variables):

\begin{eqnarray}\label{sabrder}
\frac{\partial x}{\partial K} &=& \frac{ (1-\rho)(J_y y_K + \eta y_K) }{ \eta (J + \eta y - \rho)(1-\rho) } = 
\frac{(J_y+\eta)y_K}{\eta(J+\eta y-\rho)} \\
J_y &=& \frac{2\eta^2y-2\rho\eta}{2\sqrt{1+\eta^2y^2-2\rho\eta y}} \\
y_K &=& -K^{-\beta}v^{-1}
\end{eqnarray}

Therefore

\begin{eqnarray}\label{sabrdxdk}
\frac{\partial x}{\partial K} \left( f(t) \right) = -f(t)^{-\beta} v(t)^{-1} 
\end{eqnarray}

and in the normal case

\begin{equation}
\nu(t) = f(t)^{\beta} v(t)
\end{equation}

where again $v(0)=\alpha$. In the lognormal case the rhs is replaced by $f(t)^{\beta-1}v(t)$. The same formulas hold for $\beta=1$ as well.

The equivalent deterministic local volatility function \ref{localvolfromx} is computed using the definition of $x$ and \ref{sabrder} as

\begin{equation}\label{sabrlocalvol}
- \left( \frac{\partial x}{\partial K} (f) \right)^{-1} = J(y) f^{\beta} v
\end{equation}

We note that $y$ has to be evaluated at $K=f$ here, while $f(t)$ in the definition of $y$ in \ref{defsabry} stays the forward rate at time $t=0$. Also, $v$ means $v(t=0)$ here.

\section{The ZABR model}
We generalize the SABR model by introducing a parameter $\gamma \in [0,\infty)$ and setting $\epsilon(v) = \eta v^\gamma$. For $\gamma=1$ the original
SABR model is reproduced.

An important note is that in \cite{zabr} for the model formulation $v(0)=1$ is assumed, while here we assume $v(0)=\alpha$. It is straightforward to see that the volvol parameter $\eta'$ from \cite{zabr} has to be transformed by

\begin{equation}
\eta = \eta' \alpha^{1-\gamma}
\end{equation}

to get the volvol parameter $\eta$ here.

As in the case of the SABR model we define a new variable $y$ by setting

\begin{equation}\label{defzabry}
y(t) := v(t)^{\gamma-2}\int_{K}^{f(t)} \sigma(u)^{-1} du
\end{equation}

and applying Ito to get

\begin{eqnarray}
dy = y_f df + y_v dv + \textnormal{ ... } dt = \\
v(t)^{\gamma -1} dW + (\gamma -2) \eta v(t)^{\gamma-1} y dV + \textnormal{ ... } dt
\end{eqnarray}

Setting $x(t) := v(t)^{1-\gamma}u(y(t))$ and again applying Ito yields

\begin{eqnarray}
dx = x_y dy + x_v dv + \textnormal{ ... } dt = \\
v(t)^{1-\gamma} u'(y) dy + (1-\gamma) \eta u(y) dV + \textnormal{ ... } dt = \\
u'(y) dW + ((\gamma-2)y \eta u'(y)+(1-\gamma)\eta u(y)) dV + \textnormal{ ... } dt
\end{eqnarray}

Now

\begin{eqnarray}
\frac{dx dx}{dt} = u'(y)^2 + ((\gamma-2)y \eta u'(y)+(1-\gamma)\eta u(y))^2 + \\
2 \rho u'(y) ((\gamma-2)y \eta u'(y)+(1-\gamma)\eta u(y))
\end{eqnarray}

which is equal to 1 if $u$ is a solution to the following ordinary differential equation:

\begin{eqnarray}
1 &=& A(y) u'(y)^2 + B(y) u'(y)u(y) + C u(y)^2 \\
A(y) &=& 1+(\gamma -2)^2 \eta^2 y^2 + 2\rho(\gamma-2)\eta y \\
B(y) &=& 2\rho(1-\gamma)\eta + 2(1-\gamma)(\gamma-2)\eta^2 y \\
C &=& (1-\gamma)^2\eta^2
\end{eqnarray}

subject to the initial condition $u(0)=0$. This equation can be made explicit in $u'$:

\begin{equation}\label{zabrode}
u'(y) = F(y,u(y)) := \frac{-B(y)u(y)+\sqrt{B(y)^2u(y)^2-4A(y)(Cu(y)^2-1)}}{2A(y)}
\end{equation}

( ... why do we discard the other solution here ... )

Suppose we have solved \ref{zabrode} then we can compute $x(t)$ directly from its
definition $x(t) = v(t)^{1-\gamma}u(y(t))$ for all strikes $K$ (note that $y(t)$ depends on $K$).

Also the equivalent deterministic local volatility function \ref{localvolfromx} is computed
easily  from $u(y(t))$ via

\begin{equation}
-\left(\frac{\partial x}{\partial K}(f)\right)^{-1}
\end{equation}

with

\begin{eqnarray}
\frac{\partial x}{\partial K}(f) = - v(t)^{1-\gamma} u'(y(t)) v(t)^{\gamma -2} \sigma(f)^{-1}
= \\ -v(t)^{-1}\sigma(f)^{-1} F(y,v(t)^{\gamma-1}x(t))
\end{eqnarray}

using \ref{zabrode}, which leads to

\begin{equation}\label{zabrlocalvol}
-\left(\frac{\partial x}{\partial K}(f)\right)^{-1} = v\sigma(f)F(y,v^{\gamma-1}x)^{-1}
\end{equation}

We note that the special case formula \ref{sabrdxdk} stays the same for the ZABR model as for the SABR model,
because $F(0,0) = 1$, therefore $\partial{x} / \partial{K} = -v(t)^{-1}\sigma(f)^{-1}$. Also the same remarks
concerning the evaluation of $y$ and $v$ as given for \ref{sabrlocalvol} apply here.

\section{Dupire pricing}

Given a model (which in this context here will be an equivalent deterministic local volatility model generated
from a SABR or ZABR model, see above) of the form

\begin{equation}
df(t) = \theta(f(t)) dW
\end{equation}

it is possible to use the following pde derived by Dupire to price a european call option $c(T,K)$ with strike $K$ and maturity $T$,

\begin{equation}\label{DupirePde}
\frac{\partial c}{\partial T} = \frac{\partial^2 c}{\partial K^2} \frac{\theta(K)^2}{2}
\end{equation}

with initial condition $c(0,K)=(f(0)-K)^+$. For fixed maturity $T$ this will give simultaneously call prices for all strikes $K$.

\section{Exact finite difference prices}

In this section we derive the pde to price a single european call in the ZABR model. The derivation is independent in the sense
that it does not make use of any of the approximations in the previous sections. The result can be used to produce benchmark
prices, which are exact apart from errors introduced in the numerical procedures.

Starting with definition \ref{generalmodeldef} consider a derivative price $c(t,f(t),v(t))$ which expands to

\begin{eqnarray}
dc = c_t dt + c_f df + c_v dv + \frac{1}{2} ( c_{ff} df df + c_{vv} dv dv ) + c_{fv} df dv\\
= (c_t + \frac{1}{2} \left( c_{ff}\sigma(f)^2v^2 +  c_{vv}\epsilon(v)^2 \right) + c_{fv}\sigma(f) v \epsilon(v) \rho) dt \\
+ \textnormal{ ... } dW + \textnormal{ ... } dV
\end{eqnarray}

Using $dc=0$ this yields to the pde for $c(t,f,v)$

\begin{equation}
c_t +\frac{1}{2}(c_{ff}\sigma(f)^2v^2+c_{vv}\epsilon(v)^2)+c_{fv}\sigma(f)v\epsilon(v)\rho = 0
\end{equation}

with conditions

\begin{eqnarray}
c(T,f,\cdot) = ( f - K ) ^+
\end{eqnarray}

\section{Numerical Examples}

\subsection{Classic Hagan and short maturity expansions}

We recompute the example from \cite{zabr} (page 7, figure 1) with parameters $\alpha=0.0873, \beta=0.7, \nu=0.47, \rho=-0.47$, forward
rate $f=0.0325$ (the forward is actually not mentioned in \cite{zabr}, but we believe this value was used there) and expiry time $t=10.0$. The resulting lognormal implied volatilities are displayed in figure \ref{ImplVol}. The corresponding densities are plotted in figure \ref{Dens}

\begin{figure}[htbp]
\caption{Implied Volatility, Classic Hagan formula and short maturity expansion}
\label{ImplVol}
	\begin{gnuplot}
		set terminal epslatex color
        set style line 1 linecolor rgb "blue"
        set style line 2 linecolor rgb "red"
        set style line 3 linecolor rgb "green"
        set xrange [0:0.20]
		set yrange [0:0.70]
		set xlabel "strike"
		set ylabel "implied ln vol"
        set grid
        plot 'smiles.dat' using 1:2 with lines ls 1 title "Classic Hagan",'smiles.dat' using 1:6 with lines ls 2 title "sme lognormal", 'smiles.dat' using 1:($10>0?$10:1/0) with lines ls 3 title "sme normal"
	\end{gnuplot}
\end{figure}

\begin{figure}[htbp]
\caption{Densities, Classic Hagan formula and short maturity expansion}
\label{Dens}
	\begin{gnuplot}
		set terminal epslatex color
        set style line 1 linecolor rgb "blue"
        set style line 2 linecolor rgb "red"
        set style line 3 linecolor rgb "green"
        set xrange [0:0.20]
		set yrange [-30:30]
		set xlabel "strike"
		set ylabel "density"
        set grid
        plot 'smiles.dat' using 1:5 with lines ls 1 title "Classic Hagan",'smiles.dat' using 1:9 with lines ls 2 title "sme lognormal", 'smiles.dat' using 1:($10>0?$13:1/0) with lines ls 3 title "sme normal"
	\end{gnuplot}
\end{figure}

We reproduce figure 3 from \cite{zabr}. We believe the short maturity expansion for normal volatility was used. Our result is depicted in figure \ref{Zabrgamma}

\begin{figure}[htbp]
\caption{Normal Short Maturity Expansion ZABR model with different gammas}
\label{Zabrgamma}
	\begin{gnuplot}
		set terminal epslatex colorb
        set style line 1 linecolor rgb "blue"
        set style line 2 linecolor rgb "red"
        set style line 3 linecolor rgb "green"
        set style line 4 linecolor rgb "black"
        set style line 5 linecolor rgb "orange"
        set xrange [0:0.20]
		set yrange [0:1]
		set xlabel "strike"
		set ylabel "implied lognormal vol"
        set grid
        set ytics 0,0.1,1
        plot 'smiles.dat' using 1:($10>0?$10:1/0) with lines ls 1 title "sme gamma=1.0",'smiles.dat' using 1:($20>0?$20:1/0) with lines ls 2 title "sme gamma=0.0",'smiles.dat' using 1:($21>0?$21:1/0) with lines ls 3 title "sme gamma=0.5",'smiles.dat' using 1:($22>0?$22:1/0) with lines ls 4 title "sme gamma=1.5",'smiles.dat' using 1:($23>0?$23:1/0) with lines ls 5 title "sme gamma=1.7"
	\end{gnuplot}
\end{figure}




\subsection{Equivalent deterministic local volatility}

We use \ref{sabrlocalvol} or \ref{zabrlocalvol} to compute the equivalent deterministic local volatility (with the same parameters as in the previous section). The result is displayed in figure \ref{LocalVol} which compares to \cite{zabr}, figure 2.

\begin{figure}[htbp]
\caption{Equivalent Local volatility for SABR model}
\label{LocalVol}
	\begin{gnuplot}
		set terminal epslatex color
        set style line 1 linecolor rgb "blue"
        set style line 2 linecolor rgb "red"
        set style line 3 linecolor rgb "green"
        set xrange [0:0.10]
		set yrange [0:0.045]
		set xlabel "strike k"
		set ylabel "local vol \theta(k)"
        set grid
        plot 'smiles.dat' using 1:14 with lines ls 1 title "Equivalent local vol"
	\end{gnuplot}
\end{figure}

Next we use this local vol and \ref{DupirePde} to produce a volatility smile. The result in terms of densitites is shown in \ref{LocalVolDensity} together with the classic Hagan formula implied density. Clearly the local volatility approach removes the defect of the Hagan formula for strikes near zero and produces a globally arbitrage free smile.

\begin{figure}[htbp]
\caption{Densities from Dupire pricing and Classic Hagan formula}
\label{LocalVolDensity}
	\begin{gnuplot}
		set terminal epslatex color
        set style line 1 linecolor rgb "blue"
        set style line 2 linecolor rgb "red"
        set style line 3 linecolor rgb "green"
        set xrange [0:0.20]
		set yrange [-30:30]
		set xlabel "strike"
		set ylabel "density"
        set grid
        plot 'smiles.dat' using 1:5 with lines ls 1 title "Classic Hagan", 'smiles.dat' using 1:19 with lines ls 2 title "Dupire"
	\end{gnuplot}
\end{figure}





\begin{thebibliography}{1}

\bibitem{zabr}Andreasen, Jesper and Huge, Brian: ZABR -- Expansions for the Masses, SSRN id 1980726, December 2011

\end{thebibliography}

\end{document}



