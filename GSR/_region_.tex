\message{ !name(GSR.tex)}%% Based on a TeXnicCenter-Template by Gyorgy SZEIDL.
%%%%%%%%%%%%%%%%%%%%%%%%%%%%%%%%%%%%%%%%%%%%%%%%%%%%%%%%%%%%%

%------------------------------------------------------------
%
\documentclass{amsart}
%
%----------------------------------------------------------
% This is a sample document for the AMS LaTeX Article Class
% Class options
%        -- Point size:  8pt, 9pt, 10pt (default), 11pt, 12pt
%        -- Paper size:  letterpaper(default), a4paper
%        -- Orientation: portrait(default), landscape
%        -- Print size:  oneside, twoside(default)
%        -- Quality:     final(default), draft
%        -- Title page:  notitlepage, titlepage(default)
%        -- Start chapter on left:
%                        openright(default), openany
%        -- Columns:     onecolumn(default), twocolumn
%        -- Omit extra math features:
%                        nomath
%        -- AMSfonts:    noamsfonts
%        -- PSAMSFonts  (fewer AMSfonts sizes):
%                        psamsfonts
%        -- Equation numbering:
%                        leqno(default), reqno (equation numbers are on the right side)
%        -- Equation centering:
%                        centertags(default), tbtags
%        -- Displayed equations (centered is the default):
%                        fleqn (equations start at the same distance from the right side)
%        -- Electronic journal:
%                        e-only
%------------------------------------------------------------
% For instance the command
%          \documentclass[a4paper,12pt,reqno]{amsart}
% ensures that the paper size is a4, fonts are typeset at the size 12p
% and the equation numbers are on the right side
%
\usepackage{amsmath}%
\usepackage{amsfonts}%
\usepackage{amssymb}%
\usepackage{graphicx}
%------------------------------------------------------------
% Theorem like environments
%
\newtheorem{theorem}{Theorem}
\theoremstyle{plain}
\newtheorem{acknowledgement}{Acknowledgement}
\newtheorem{algorithm}{Algorithm}
\newtheorem{axiom}{Axiom}
\newtheorem{case}{Case}
\newtheorem{claim}{Claim}
\newtheorem{conclusion}{Conclusion}
\newtheorem{condition}{Condition}
\newtheorem{conjecture}{Conjecture}
\newtheorem{corollary}{Corollary}
\newtheorem{criterion}{Criterion}
\newtheorem{definition}{Definition}
\newtheorem{example}{Example}
\newtheorem{exercise}{Exercise}
\newtheorem{lemma}{Lemma}
\newtheorem{notation}{Notation}
\newtheorem{problem}{Problem}
\newtheorem{proposition}{Proposition}
\newtheorem{remark}{Remark}
\newtheorem{solution}{Solution}
\newtheorem{summary}{Summary}
\numberwithin{equation}{section}
%--------------------------------------------------------
\begin{document}

\message{ !name(GSR.tex) !offset(-3) }

\title[One Factor Gaussian Short Rate Model Implementation]{One Factor Gaussian Short Rate Model Implementation}
\author{P. Caspers}
\email[P. Caspers]{pcaspers1973@googlemail.com}
\date{April 29, 2013}
\dedicatory{First Version March 1, 2013 - This Version April 29, 2013}
\begin{abstract}
We collect some results in Piterbarg, Interest Rate Modelling, needed for the implementation of a GSR model. We develop explicit formulas
for piecewise constant volatility and reversion parameters under the forward measure.
\end{abstract}

\maketitle


\section{model}
The short rate dynamics is given by

\begin{equation}
dr(t) = \kappa(t) ( \theta(t) - r(t) ) dt + \sigma_r(t) dW(t)
\end{equation}

under the risk neutral measure. $\kappa, \sigma_r$ are piecewise constant. Setting $x(t) := r(t) - f(0,t)$ with $f(t,T)$ denoting
the instanteous forward rate observed at $t$ for $T>t$, the dynamics can be rewritten

\begin{equation}
dx(t) = ( y(t) - \kappa(t) x(t) ) dt + \sigma_r(t) dW(t)
\end{equation}

with deterministic

\begin{equation}\label{smallY}
y(t) = \int_0^t e^{-2 \int_u^t \kappa(s) ds} \sigma_r(u)^2 du
\end{equation}

In the $T-$forward measure the dynamics becomes

\begin{equation}
dx(t) = (y(t)-\sigma_r(t)^2G(t,T)-\kappa(t)x(t))dt+\sigma_r(t)dW^T(t)
\end{equation}

with

\begin{equation}\label{bigG}
G(t,t') = \int_t^{t'} e^{-\int_t^u\kappa(s)ds} du
\end{equation}

This fits into the general treatment under \ref{generalSDE} with

\begin{eqnarray}
a(t) &=& -\kappa(t) \\
b(t) &=& y(t)-\sigma_r(t)^2G(t,T) \\
c(t) &=& \sigma_r(t)
\end{eqnarray}

Zero bond prices can be expressed as follows

\begin{equation}
P(t,t') = \frac{P(0,t')}{P(0,t)} e^{-x(t)G(t,t')-\frac{1}{2}y(t)G(t,t')^2}
\end{equation}

\section{Basic SDE Integration}
Consider the SDE

\begin{equation}\label{generalSDE}
dx(t) = (a(t)x(t) + b(t)) dt + c(t) dW(t)
\end{equation}

with deterministic scalar functions $a, b, c$. The following is an explicit solution of \ref{generalSDE}.

\begin{equation}
x(t) = e^{\int_0^t a(u) du} \left( x(0)  + \int_0^t e^{-\int_0^s a(u) du} b(s) ds + \int_0^t e^{-\int_0^s a(u) du} c(s) dW(s) \right)
\end{equation}

That means that for $w<t$ $x(t)$ conditional on $x(w)$ is normally distributed with mean and variance given by

\begin{eqnarray}\label{condExpectationVariance}
E(x(t) | x(w)) = A(w,t) x(w) + \int_{w}^t A(s,t) b(s) ds  \\
\textnormal{Var}(x(t) | x(w)) = \int_{w}^t A(s,t)^{2} c(s)^2 ds
\end{eqnarray}

with short hand notation

\begin{equation}
A(s,t) = e^{\int_s^t a(u) du}
\end{equation}

\section{Formulas for the piecewise constant case}

Under the assumption of piecewise constant $\kappa, \sigma_r$ we are left with the computation of some integrals for which we derive
closed form formulas here. We fix a grid $0=t_0 < t_1 < ... < t_n =T$ such that $\kappa$ and $\sigma_r$ are constant on each interval $[t_i,t_{i+1})$ and equal to $\kappa_i, \sigma_{r,i}$. We introduce the following notation: $l(t)$ denotes the largest index such that $t_{l(t)} \leq t$. Likewise $h(t)$ denotes the smallest index such that $t_{h(t)}\geq t$. Moreover we set $t_{i,s} := \max( t_i, s )$ and $t_{s,i} := \min( t_i, s)$.

\subsection{Formula for $G(t,t')$}

We start with \ref{bigG}, the formula for $G(t,t')$. The integrand is

\begin{equation}
e^{-\int_t^u \kappa(s) ds} = \prod_{i=l(t)}^{h(u)-1} e^{-\kappa_i\int_{t_{i,t}}^{t_{u,i+1}} ds } = \prod_{i=l(t)}^{h(u)-1} e^{-\kappa_i(t_{u,i+1}-t_{i,t})}
\end{equation}

By this

\begin{equation}
G(t,t') = \sum_{i=l(t)}^{h(t')-1} \int_{t_{i,t}}^{t_{t',i+1}} \left( \prod_{j=l(t)}^{i-1} e^{-\kappa_j(t_{j+1}-t_{j,t})} \right) e^{-\kappa_i(u-t_{i,t})} du
\end{equation}

which is

\begin{equation}
\sum_{i=l(t)}^{h(t')-1} \left( \frac{1- e^{-\kappa_i(t_{t',i+1}-t_{i,t})}}{\kappa_i} \right) \prod_{j=l(t)}^{i-1} e^{-\kappa_j(t_{j+1}-t_{j,t})} 
\end{equation}

We abbreviate this by

\begin{equation}
G(t,t') = \sum_{i=l(t)}^{h(t')-1} \eta_i \prod_{j=l(t)}^{i-1} \gamma_{j} 
\end{equation}

$\gamma_j$ is dependent on $t$ if and only if $j=l(t)$. In this case $i>l(t)$ necessarily. $\eta_i$ is dependent on $t$ if and only if $i=l(t)$. In these cases

\begin{eqnarray}
\int \gamma_{j} dt &=& \frac{e^{-\kappa_j(t_{j+1}-t)}}{\kappa_j} \\
\int \eta_i dt &=& \frac{t\kappa_i-e^{-\kappa_i(t_{t',i+1}-t)}}{\kappa_i^2}
\end{eqnarray} 

If $j>l(t)$ resp. $i>l(t)$ these integrals can be computed trivially by multiplying $\gamma_j$ resp. $\eta_i$ by the interval length over which the integral is computed.

\subsection{Formula for $y(t)$}

We continue with \ref{smallY}. The integrand here is

\begin{equation}
e^{-2 \int_u^t \kappa(s) ds} \sigma_r(u)^2 = \prod_{i=l(u)}^{h(t)-1} \sigma_{r,i}^2 e^{-2\kappa_i(t_{t,i+1}-t_{i,u})}
\end{equation}

We get

\begin{equation}
y(t) = \sum_{i=0}^{h(t)-1} \int_{t_i}^{t_{t,i+1}} \left( e^{-2\kappa_i(t_{t,i+1}-u)} \prod_{j=i+1}^{h(t)-1} \sigma_{r,j}^2 e^{-2\kappa_j(t_{t,j+1}-t_j)} \right) du
\end{equation}

which is

\begin{equation}\label{explicitY}
\sum_{i=0}^{h(t)-1} \left( \frac{\sigma_ {r,i}^2}{2\kappa_i} \left[1-e^{-2\kappa_i(t_{t,i+1}-t_i)}\right] \prod_{j=i+1}^{h(t)-1} e^{-2\kappa_j(t_{t,j+1}-t_j)} \right)
\end{equation}

and which we abbreviate by

\begin{equation}
y(t) = \sum_{i=0}^{h(t)-1} \left( \alpha_i \prod_{j=i+1}^{h(t)-1} \beta_j \right)
\end{equation}

$\alpha_i$ resp. $\beta_j$ is dependent on $t$ if and only if $i=h(t)-1$ resp. $j=h(t)-1$. In the latter case $i<h(t)-1$ necessarily. In these cases

\begin{eqnarray}
\int \alpha_i dt = \frac{\sigma^2_{r,i}(t\kappa_i + e^{-2\kappa_i(t-t_i)})}{2\kappa_i^2}  \\
\int \beta_j dt = - \frac {e^{-2\kappa_j(t-t_j)}} {2\kappa_j} 
\end{eqnarray}

If $i<h(t)-1$ resp. $j<h(t)-1$ these integrals can trivially computed by multiplying $\alpha_i$ resp. $\beta_j$ by the interval length over which the 
integral is computed.

\subsection{Formula for $A(s,t)$}

Now we continue with the formulas for conditional expectation and variance \ref{condExpectationVariance}. First of all we notice that

\begin{equation}
A(s,t) = \prod_{i=l(s)}^{h(t)-1} e^{-\kappa_i(t_{t,i+1}-t_{i,s})} = \prod_{i=l(s)}^{h(t)-1} \zeta_i
\end{equation}

$\zeta_i$ is dependent on $s$ if and only if $i=l(s)$. In this case

\begin{equation}
\int \zeta_i ds = \frac{1}{\kappa_i}e^{-\kappa_i(t_{t,i+1}-s)}
\end{equation}

If $i>l(s)$ the integral can be computed trivially by multiplying $\zeta_i$ by the interval length over which the integral is computed.

\subsection{Formula for $E(x(t)|x(w))$}

\subsubsection{The easy part}

The first term $A(w,t) x(w)$ in the conditional expectation can easily be computed with the result obtained so far. The second term is

\begin{equation}
\int_{w}^t A(s,t) ( y(s)-\sigma_r(s)^2G(s,T))  ds
\end{equation}

\subsubsection{First not so easy part of the integral}

Let's start with the integral over $A(s,t)y(s)$, which is

\begin{equation}
\sum_{k=l(w)}^{h(t)-1} \int_{t_{k,w}}^{t_{t,k+1}} \sum_{l=0}^{k} \alpha_l  \left( \prod_{i=k}^{h(t)-1} \zeta_i \prod_{j=l+1}^{k} \beta_j \right) ds
\end{equation}

We integrate each single summand, i.e. we fix $k$ and $l$. $\zeta_i$ depends on $s$ iff $i=k$. $beta$ depends on $s$ iff $j=k$. $\alpha_l$ depends
on $s$ iff $l=k$.

Consider the case $l<k$ first. Then $\alpha_l$ does not depend on $s$. Amongst the factors in round brackets exactly $\zeta_k$ and $\beta_k$ are depending on $s$. In essence we are left with computation of

\begin{equation}
\int \zeta_k \beta_k ds
\end{equation}

which is

\begin{equation}
\int e^{-\kappa_k(t_{t,k+1}-s)} e^{-2\kappa_k(s-t_k)} ds 
\end{equation}

This again is explicitly

\begin{equation}
-\frac{1}{\kappa_k}e^{-\kappa_ks+\kappa_k(2t_k-t_{t,k+1})}
\end{equation}

Now consider the case $l=k$. In this case exactly the factors $alpha_k$ and $\zeta_k$ depend on $s$. Note that $\beta_k$ does not occur in the product in this case. We have therefore to evaluate

\begin{equation}
\int \zeta_k \alpha_k ds = \int e^{-\kappa_k(t_{t,k+1}-s)} \frac{\sigma_{r,k}^2}{2\kappa_k}\left[1-e^{-2\kappa_k(s-t_k)}\right] ds
\end{equation}

This simplifies to

\begin{equation}
\frac{\sigma_{r,k}^2}{2\kappa_k} \int e^{-\kappa_k(t_{t,k+1}-s)} - e^{-\kappa_k s + \kappa_k ( 2t_k - t_{t,k+1} ) } ds
\end{equation}

which is in explicit terms

\begin{equation}
\frac{\sigma_{r,k}^2}{2\kappa_k^2} \left( e^{-\kappa_k s + \kappa_k ( 2t_k - t_{t,k+1} ) } + e^{-\kappa_k(t_{t,k+1}-s)} \right)
\end{equation}

\subsubsection{Second not so easy part of the integral} 

Similarly the integral over $-A(s,t)\sigma_r(s)^2G(s,T)$ can be written

\begin{equation}
-\sum_{k=l(w)}^{h(t)-1} \sigma_k^2 \int_{t_{k,w}}^{t_{t,k+1}} \sum_{l=k}^{h(T)-1} \eta_l \left( \prod_{i=k}^{h(t)-1} \zeta_i   
   \prod_{j=k}^{l-1} \gamma_j  \right) ds
\end{equation}

As above, fix $k$ and $l$. $\eta_l$ is dependent on $s$ iff $l=k$, $\zeta_i$ is dependent on $s$ iff $i=k$, $\gamma_j$ is dependent on $s$ iff $j=k$. 

Again we start wit the case $l>k$. As above we are left with $\int \zeta_k \gamma_k$, which is

\begin{equation}
\int e^{-\kappa_k(t_{t,k+1}-s)} e^{-\kappa_k(t_{k+1}-s)} ds
\end{equation}

and explicitly

\begin{equation}
\frac{e^{2\kappa_ks-\kappa_k(t_{t,k+1}+t_{k+1})}}{2\kappa_k}
\end{equation}

If on the other hand $l=k$, we face $\int \eta_k\zeta_k$ which can be computed as

\begin{equation}
\int e^{-\kappa_k(t_{t,k+1}-s)} \left( \frac{1- e^{-\kappa_k(t_{T,k+1}-s)}}{\kappa_k} \right) ds
\end{equation}

and further 

\begin{equation}
\frac{2e^{-\kappa_k(t_{t,k+1}-s)} - e^{2\kappa_ks -\kappa_k(t_{T,k+1}+t_{t,k+1})}} {2\kappa_k^2}
\end{equation}

\subsection{Formula for $\textnormal{Var}(x(t) | x(w)) $}
Finally we analyze the integral representing the conditional variance, which is

\begin{equation}
\int_w^t A(s,t)^2 \sigma_r(t)^2 ds
\end{equation}

As before we write

\begin{equation}
\sum_{k=l(w)}^{h(t)-1} \sigma_{r,k}^2 \int_{t_{k,w}}^{t_{t,k+1}} \prod_{i=k}^{h(t)-1} \zeta_i^2 ds
\end{equation}

For fixed $k$ the term not covered yet is $\int \zeta_k^2 ds$, which is

\begin{equation}
\int e^{-2\kappa_k(t_{t,k+1}-s)} ds = \frac{e^{2\kappa_k(s-t_{t,k+1})} }{2\kappa_k}
\end{equation}

\end{document}
\message{ !name(GSR.tex) !offset(-363) }
