%% Based on a TeXnicCenter-Template by Gyorgy SZEIDL.
%%%%%%%%%%%%%%%%%%%%%%%%%%%%%%%%%%%%%%%%%%%%%%%%%%%%%%%%%%%%%

%------------------------------------------------------------
%
\documentclass{amsart}
%
%----------------------------------------------------------
% This is a sample document for the AMS LaTeX Article Class
% Class options
%        -- Point size:  8pt, 9pt, 10pt (default), 11pt, 12pt
%        -- Paper size:  letterpaper(default), a4paper
%        -- Orientation: portrait(default), landscape
%        -- Print size:  oneside, twoside(default)
%        -- Quality:     final(default), draft
%        -- Title page:  notitlepage, titlepage(default)
%        -- Start chapter on left:
%                        openright(default), openany
%        -- Columns:     onecolumn(default), twocolumn
%        -- Omit extra math features:
%                        nomath
%        -- AMSfonts:    noamsfonts
%        -- PSAMSFonts  (fewer AMSfonts sizes):
%                        psamsfonts
%        -- Equation numbering:
%                        leqno(default), reqno (equation numbers are on the right side)
%        -- Equation centering:
%                        centertags(default), tbtags
%        -- Displayed equations (centered is the default):
%                        fleqn (equations start at the same distance from the right side)
%        -- Electronic journal:
%                        e-only
%------------------------------------------------------------
% For instance the command
%          \documentclass[a4paper,12pt,reqno]{amsart}
% ensures that the paper size is a4, fonts are typeset at the size 12p
% and the equation numbers are on the right side
%
\usepackage{amsmath}%
\usepackage{amsfonts}%
\usepackage{amssymb}%
\usepackage{graphicx}
\usepackage{gnuplottex}
\usepackage{epstopdf}
%\usepackage[pdf]{pstricks}
%------------------------------------------------------------
% Theorem like environments
%
\newtheorem{theorem}{Theorem}
\theoremstyle{plain}
\newtheorem{acknowledgement}{Acknowledgement}
\newtheorem{algorithm}{Algorithm}
\newtheorem{axiom}{Axiom}
\newtheorem{case}{Case}
\newtheorem{claim}{Claim}
\newtheorem{conclusion}{Conclusion}
\newtheorem{condition}{Condition}
\newtheorem{conjecture}{Conjecture}
\newtheorem{corollary}{Corollary}
\newtheorem{criterion}{Criterion}
\newtheorem{definition}{Definition}
\newtheorem{example}{Example}
\newtheorem{exercise}{Exercise}
\newtheorem{lemma}{Lemma}
\newtheorem{notation}{Notation}
\newtheorem{problem}{Problem}
\newtheorem{proposition}{Proposition}
\newtheorem{remark}{Remark}
\newtheorem{solution}{Solution}
\newtheorem{summary}{Summary}
\numberwithin{equation}{section}
%--------------------------------------------------------
\begin{document}
\title[FX Basics]{FX Basics (DRAFT VERSION)}
\author{P. Caspers}
\email[P. Caspers]{pcaspers1973@googlemail.com}
\date{July 23, 2013}
\dedicatory{First Version July 23, 2013 - This Version July 23, 2013}
\begin{abstract}
We summarize some basics from FX Derivative pricing with references to \cite{Clark} and the notation
in the Murex system.
\end{abstract}

\maketitle

\section{Option Price Quotations}

We consider a currency pair ccy1/ccy2 with ccy1 denoting the foreign (base, asset) and ccy2 the domestic (underlying, numeraire) currency. 
To fix an example setup we choose ccy1 to be EUR and ccy2 to be USD. Now consider a european option to exchange an amount $N_d$ in domestic
currecy with an amount $N_f$ in foreign currency. The strike $K$ of the option is by definition $K=N_d / N_f$. If for
example $N_d = 132mm$ USD and $N_f = 100mm$ EUR then $K = 1.32$. We consider different possible price quotations.

\subsection{Domestic pips $V_{\textnormal{d, pips}} = V_{\textnormal{d/f}}$ or ccy2 pips}

The price is expressed as the option premium in domestic currency per unit amount in foreign currency. In the Murex screen
we might see $449.4077$ USD pips which means that the premium to be paid in cash is $4,494,077$ USD (multiply the USD pips
quotation by the EUR nominal, where one has to know in addition that $1$ pip = $0.01\%$ = $0.0001$).

The notation in \cite{Clark} is $V_{\textnormal{d,pips}}$ or $V_{\textnormal{d/f}}$, in Murex it is ccy2 pips.

\subsection{Foreign percent $V_{\%\textnormal{f}}$ or \% ccy2}

The price is expressed as the option premium in foreign currency per unit amount in foreign currency. If $S$ is the
FX exchange rate (i.e. the value of one unit of foreign currency in domestic currency, e.g. the value of $1$ EUR in USD in
our example setting) applicable to the settlement of the premium payment, we can obviously compute this quantity as
$V_{\%\textnormal{f}} = V_{\textnormal{d,pips}} / S$. If for example $S = 1.3172$ we would expect $V_{\%\textnormal{f}} = 3.411841$ in our example above. In
Murex we would see this quantity as $3.411841$ \% EUR.

The notation in \cite{Clark} is $V_{\%\textnormal{f}}$, in Murex it is \% ccy1.


\subsection{Domestic percent $V_{\%\textnormal{d}}$ or \% ccy1}.

The price is expressed as the option premium in domestic currency per unit amount in domestic currency. Since
$N_d = K N_f$ we can directy express this quantity as $V_{\textnormal{d, pips}} / K$. In our example setting we get
$V_{\%\textnormal{d}} = 340.4604$ (since $K=1.32$ and $V_{\textnormal{d, pips}} = 449.4077$) which is displayed in Murex as 
$3.404604$ \% USD.

The notation in \cite{Clark} is $V_{\%\textnormal{d}}$, in Murex it is \% ccy2.

\subsection{Foreign pips $V_{\textnormal{f, pips}} = V_{\textnormal{f/d}}$ or ccy1 pips.}

The price is expressed as the option premium in foreign currency per unit amount in domestic currency. This can be
easily derived as $V_{\%\textnormal{d}} / S$, with above figures this yields $258.4728$ EUR pips in Murex notation.

The notation in \cite{Clark} is $V_{\textnormal{f, pips}}$, in Murex it is ccy1 pips.

\subsection{Absolute domestic premium $V_d$}

The price is given as the absolute premium in domestic currency $V_d = N_f V_{d/f}$, which is $4,494,077$ USD in
the example here.

The notation in \cite{Clark} is $V_d$, in Murex this value can be seen in the field price amount (with the price
field set to a ccy2 currency quotation mode).

\subsection{Absolute foreign premium $V_f$}

The price is given as the absolute premium in foreign currency $V_f = N_d V_{f/d}$, which is $3,411,841$ EUR in
our example, easily computed from the figures above.

The notation in \cite{Clark} is $V_f$, in Murex this value can again be seen in the field price amount (with the
price field set to a ccy1 currency quotation mode).

\section{Option Delta Quotation}

\subsection{Pips spot delta or \% ccy1}
\label{pipsDelta}

The pips spot delta according to \cite{Clark}, 3.2.1 is defined as

\begin{equation}
\Delta_{S\textnormal{; pips}} = \frac{\partial V_{\textnormal{d,pips}}}{\partial S}
\end{equation}

i.e. the change in the domestic pips premium (see above) when the spot $S$ changes by an infinitesimal amonut.

For a typical example like an EUR/USD call (i.e. call on EUR, put on USD) in Murex we see this delta displayed
as \% EUR, e.g. $53.957408$ \% EUR. This delta can be understood as the EUR amount (given as percentage of the
EUR nominal) needed for a delta hedge of the option.

The above value is displayed in Murex only if the premium quotation is one of the USD quotations (\% USD or
USD pips). If the premium on the other hand is quoted as \% EUR or EUR pips the \% EUR delta in Murex is
the percentage spot delta or premium adjusted pips delta in the sense of \cite{Clark}, 3.2.2 or as described
in section \ref{percDelta} of this paper.

This is intuitive since if the premium is paid in EUR (the foreign currency) from the domestic currency
point of view (USD) the premium has a fx risk itself. For example, if you as an USD based bank buy an
EUR/USD call on $100$ mm EUR notional the hedge would be to sell $53,957,408$ EUR in the example above.
This is true when the premium you have to pay to purchase the option is paid in USD. If on the other
hand the premium is paid in EUR, you have to sell an EUR amount adjusted downwards exactly by the EUR premium.
If the premium is e.g. $3.733845$ \% EUR, the the hedge would be to sell $50,223,562$ EUR.

This is exactly the definition of the premium adjusted pips delta in \ref{percDelta}, here it is 
$50.223562$ \% EUR.

In general in Murex you get the pips spot delta in Murex as \% ccy1, when the premium quotation is set
to \% ccy2 or ccy2 pips and the percentage spot delta when the premium quotation is set to \% ccy1
or ccy1 pips.

There is another delta available in Murex labeled \% ccy2. This delta refers to flipping ccy1 and ccy2,
i.e. it is the delta from an ccy1 based bank. In our example above the \% USD delta would be the pips
spot delta in an USD-EUR world (USD is the foreign, EUR is domestic currency), when the premium is paid
in EUR, and the percentage spot delta when the premium is paid in USD.

In general you get the flipped pips spot delta as \% ccy2, when the premium quotation is set to \% ccy1
or ccy1 pips, and the flipped percentage spot delta when the premium quotation is set to \% ccy2 or
ccy2 pips2.


\subsection{Percentage spot delta or premium adjusted pips delta}
\label{percDelta}

The percentage spot delta according to \cite{Clark}, 3.2.2 is given by

\begin{equation}
\Delta_{S; \%} = \frac{\partial V_{\%f}}{\partial \ln S} = \Delta_{S\textnormal{; pips}} - V_{\% f}
\end{equation}

i.e. it is the premium adjusted pips delta. The economic interpretation was already discussed above
in section \ref{pipsDelta} and also under which circumstances this delta is displayed in Murex.

\subsection{Forward delta}
\label{forwardDelta}

There are forward delta flavours both of the pips delta and percentage deltas above, which involve
the forward price and the forward fx in the numerator and denominator of the definitions under
\ref{pipsDelta} and \ref{percDelta}. See also \cite{Clark}, 3.2.3 amd 3.2.4.

To compute the forward deltas from the spot deltas above, you simply apply the foreign compounding
factor $e^{r_fT}$ to the spot formulas, where $r_f$ is the foreign interest rate for maturity $T$.

Clark notes that there is another notion of forward delta involving the spot price and the forward
fx. Obviously in Murex the former definition is adopted.

\subsection{Premium in Delta setting in Murex}
\label{PremiumInDeltaMx}

It is worthwhile to note that the dynamic display of pips or percentage delta depending on the premium
currency is linked to a specific setting in Pricing Setup / Premium in delta (which can be done
locally for a user or globally). If unticked always the pips delta is displayed, no matter what premium
quotation is chosen. The default setting is to include the premium into the delta (ticked option).

Since the premium in delta choice also affects the quotation of volatilities (since they are given
in terms of delta, not strike), there is another setting in the OTC contract description / Volatility
convention / Premium in delta. If ticked, the premium will be assumed to be paid in foreign currency
ccy1 and included in the delta of the volatility quotation. For example, for EUR/USD the premium is not
included (since paid in USD by market convention), for EUR/SEK it is included (since paid in EUR by market
convention).

\section{ATM Definitions}
\label{atmdef}
\subsection{ATM Forward}

The atm level is by definition the forward fx rate, see also \cite{Clark}, 3.5.1. In Murex this atm level 
can be entered as FWD and is displayed as ATM FWD afterwards.

\subsection{Delta neutral straddle}

The atm level is by definition the strike where the delta of a call and a put is equal (i.e. a
straddle w.r.t. this strike has a zero delta). Since spot deltas and forward deltas only differ
by a constant factor (the compound factor in foreign currency, see \ref{forwardDelta}), it is
irrelvant for this definition which delta we choose. It makes a difference however, if we
take the pips or percentage delta. The convention here is the same as for the deltas itself
(see \ref{PremiumInDeltaMx}).

Explicit Formulas for the delta neutral straddle atm level are given in \cite{Clark}, 3.5.2. It
should be noted that this strike depends on the atm volatility.

In Murex this ATM Level can be entered as ATM DNS and is displayed as ATM D.NEUTR. afterwards.


\section{Risk Reversal, Strangles}

This section follows \cite{Clark}, 3.8.

\subsection{Risk reversal and Smile Strangle}

A risk reversal is a long call and a short put, both with a strike corresponding to $x\Delta$,
usually $25\Delta$ and $10\Delta$. Note that the Delta used here (spot or forward and pips or
percentage) has to be specified to get a unique definition.

The quotation of the risk reversal is done in difference of implied volatilities of single
calls and puts, i.e. for $25\Delta$

\begin{equation}\label{eqRR}
\sigma_{25\Delta\textnormal{ RR}} = \sigma_{25\Delta\textnormal{ ccy1 Call}} - \sigma_{25\Delta\textnormal{ ccy1 Put}}
\end{equation}

for a ``positive risk reversal in favor of a ccy1 call'', if the implied volatility of the call
is higher than the implied volatility of the put.

This quotation is not enough to price the risk reversal nor the single call or put. In addition
we need the atm implied volatility (again atm needs to be speficied to be ATMF or ATMDNS, see
\ref{atmdef}) as well as a smile strangle quotation, which is

\begin{equation}\label{eqSS}
\sigma_{25\Delta\textnormal{ SS}} = \frac{1}{2} ( \sigma_{25\Delta\textnormal{ ccy1 Call}} + \sigma_{25\Delta\textnormal{ ccy1 Put}} ) - \sigma_{\textnormal{atm}}
\end{equation}

(Note that there is not need to specify the currency ccy1 or ccy2 here, because a $25\Delta$ ccy1 call
is a $25\Delta$ ccy2 put and vice versa (of course the convention for the $\Delta$ kept fix)).

From equations \ref{eqRR} and \ref{eqSS} by simple algebra we get

\begin{eqnarray}
\sigma_{25\Delta\textnormal{ ccy1 Call}} = \sigma_{25\Delta\textnormal{ atm}} + \sigma_{25\Delta\textnormal{ SS}} + \frac{1}{2} \sigma_{25\Delta\textnormal{ RR}} \\
\sigma_{25\Delta\textnormal{ ccy1 Call}} = \sigma_{25\Delta\textnormal{ atm}} + \sigma_{25\Delta\textnormal{ SS}} - \frac{1}{2} \sigma_{25\Delta\textnormal{ RR}}
\end{eqnarray}

\subsection{Market Strangle}

The situation becomes more complicated if one wants to use a market strangle quote instead of a 
smile strangle quote. This is by defintion one implied volatility quote that can be added to the 
quoted atm volatility as a single volatility input to the black scholes formulas for call and put
to retrieve the strangle price. Also the strikes of call and put are determined using this single
volatility by matching the respective delta (e.g. $25\Delta$).

To convert the market strangle quote to a smile strangle quote one needs assumptions on the 
volatility smile curve parametrization. Details can be found in \cite{Clark}, 3.7. We do not
go into further details here.

\subsection{Murex Configuration}

The murex smile configuration is done under Market Data / Currency Pairs / OTC contract description.
Under volatility conventions one can specify the following parameters.

\subsubsection{Smile for a call on}
ccy1 or ccy2 specifies how a call volatility is to be read.

\subsubsection{Premium in delta}
this was already mentioned in \ref{PremiumInDeltaMx}. If yes, the premium in foreign currency will
be included in the delta.

\subsubsection{Central point for short term / long term}
two maturity regimes can be defined with different atm conventions, which may be atm delta
neutral straddle or atm forward.

\subsubsection{Spot delta / Forward delta}
again, two maturity regimes can be defined with different delta conventions (spot or forward
delta).

\subsubsection{Broker / Smile mode}
if smile is chosen, a smile strangle quote is assumed, otherwise (broker) a market strangle
quote.

\subsubsection{Positive RR in favor of}
call or put, i.e. if the RR quote is positive call (put) the implied volatility is higher than put
(call).



\begin{thebibliography}{4}

\bibitem{Clark}Clark: Foreign exchange option pricing, A Practioner's Guide

\end{thebibliography}
  

\end{document}
