%% Based on a TeXnicCenter-Template by Gyorgy SZEIDL.
%%%%%%%%%%%%%%%%%%%%%%%%%%%%%%%%%%%%%%%%%%%%%%%%%%%%%%%%%%%%%

%------------------------------------------------------------
%
\documentclass{amsart}
%
%----------------------------------------------------------
% This is a sample document for the AMS LaTeX Article Class
% Class options
%        -- Point size:  8pt, 9pt, 10pt (default), 11pt, 12pt
%        -- Paper size:  letterpaper(default), a4paper
%        -- Orientation: portrait(default), landscape
%        -- Print size:  oneside, twoside(default)
%        -- Quality:     final(default), draft
%        -- Title page:  notitlepage, titlepage(default)
%        -- Start chapter on left:
%                        openright(default), openany
%        -- Columns:     onecolumn(default), twocolumn
%        -- Omit extra math features:
%                        nomath
%        -- AMSfonts:    noamsfonts
%        -- PSAMSFonts  (fewer AMSfonts sizes):
%                        psamsfonts
%        -- Equation numbering:
%                        leqno(default), reqno (equation numbers are on the right side)
%        -- Equation centering:
%                        centertags(default), tbtags
%        -- Displayed equations (centered is the default):
%                        fleqn (equations start at the same distance from the right side)
%        -- Electronic journal:
%                        e-only
%------------------------------------------------------------
% For instance the command
%          \documentclass[a4paper,12pt,reqno]{amsart}
% ensures that the paper size is a4, fonts are typeset at the size 12p
% and the equation numbers are on the right side
%
\usepackage{amsmath}%
\usepackage{amsfonts}%
\usepackage{amssymb}%
\usepackage{graphicx}
\usepackage{gnuplottex}
\usepackage{epstopdf}
%\usepackage[pdf]{pstricks}
%------------------------------------------------------------
% Theorem like environments
%
\newtheorem{theorem}{Theorem}
\theoremstyle{plain}
\newtheorem{acknowledgement}{Acknowledgement}
\newtheorem{algorithm}{Algorithm}
\newtheorem{axiom}{Axiom}
\newtheorem{case}{Case}
\newtheorem{claim}{Claim}
\newtheorem{conclusion}{Conclusion}
\newtheorem{condition}{Condition}
\newtheorem{conjecture}{Conjecture}
\newtheorem{corollary}{Corollary}
\newtheorem{criterion}{Criterion}
\newtheorem{definition}{Definition}
\newtheorem{example}{Example}
\newtheorem{exercise}{Exercise}
\newtheorem{lemma}{Lemma}
\newtheorem{notation}{Notation}
\newtheorem{problem}{Problem}
\newtheorem{proposition}{Proposition}
\newtheorem{remark}{Remark}
\newtheorem{solution}{Solution}
\newtheorem{summary}{Summary}
\numberwithin{equation}{section}
%-----------------------------------------------------------------
\begin{document}

\title{Modelling Volatility Smiles with Spline Densities}
\author{Peter Caspers}
\date{September 18, 2013}
\dedicatory{First Version September 18, 2013 - This Version September 18, 2013}
\maketitle

\begin{abstract}
todo...
\end{abstract}

\section{Method}

We assume a strike grid $l=k_0 < ... < k_n=u$ and arbitrage free, non discounted call prices $f-l = c_0 < ... < c_n = 0$, where $f$ denotes the underlying forward. We aim to construct a density function $\phi$ which is smooth and matches the call prices exactly. The idea is to use a cubic spline within $[l,u]$ using the given strike grid and set the density zero outside this interval. As additional conditions we set $\phi'(l) = \phi'(u) = 0$, which ensures that the density is $C^1(\mathcal{R})$ globally on the real line and $C^2(l,u)$.

Denoting the call price function by $c:\mathcal{R}\rightarrow[0,f]$ we have the relation

\begin{equation}
\frac{\partial^2 c}{\partial k^2} (t) = \phi(t)
\end{equation}

for the density, as well as

\begin{equation}
\frac{\partial c}{\partial k} (t) = \int_{l}^{u} \phi(s) ds - 1
\end{equation}

for the distribution / negative digital prices and finally

\begin{equation}\label{integralC}
c(t) = f - l + \int_{l}^{t} \left( \int_{l}^{v} \phi(s) ds -1 \right) dv
\end{equation}

for the call price function itself.

The density $\phi$ is given by cubic polynomials

\begin{equation}
p_i(s) = a_i (s-k_i)^3 + b_i (s-k_i)^2 + c_i (s-k_i) + d_i
\end{equation}

on each interval $[k_i,k_{i+1}], i=0,...,n-1$. We start with the computation of the inner integral in \ref{integralC}. For each $i$ we can compute

\begin{equation}\label{intpi}
\int_{k_i}^{v} p_i(s) ds = \frac{1}{4}a_i(v-k_i)^4 + \frac{1}{3} b_i (v-k_i)^3 + \frac{1}{2} c_i (v-k_i)^2 + d(v-k_i)
\end{equation}

For the integral over a full subinterval we introduce the notation

\begin{equation}\label{intpifull}
\int_{k_i}^{k_{i+1}} p_i(s) ds = \eta_{i}
\end{equation}

We continue with the outer integration in \ref{integralC}. We write $c(t)-f$ as

\begin{equation}
\int_{l}^{t} \left[ \left( \sum_{i=0}^{m(v)-1} \int_{k_i}^{k_{i+1}} \phi(s) ds \right) + \int_{k_{m(t)}}^v \phi(s) ds - 1 \right] dv
\end{equation}

with $m(t)$ such that $t \in [k_{m(t)} , k_{m(t)+1} )$. The left integral is independent of $v$ and was computed in \ref{intpifull}.  For the right integral we have a closed form expression \ref{intpi}. Therefore we can continue to write

\begin{eqnarray}
\int_{l}^{t} \left[ \left( \sum_{i=0}^{m(v)-1} \eta_i \right) + \frac{1}{4}a_{m(v)}(v-k_{m(v)})^4 + ... - 1 \right] dv
\end{eqnarray}

We decompose the outer integral as well on our strike grid and get 

\begin{equation}
\sum_{j=0}^{m(t)} \int_{k_j}^{\min(k_{j+1},t)} \left[\left( \sum_{i=0}^{j-1} \eta_i \right) + \frac{1}{4}a_{j}(v-k_{j})^4 + ... -1 \right] dv
\end{equation}

which leads to our final formula

\begin{equation}
c(t) = f - l + \sum_{j=0}^{m(t)} \left\{ (\mu_j-k_j) \left( \lambda_j -1 \right) + \frac{1}{20}a_j(\mu_j-k_j)^5 + ... + \frac{1}{2}d_j(\mu_j-k_j)^2 \right\}
\end{equation}

with $\mu_j := \min(k_{j+1},t)$ and $\lambda_j := \sum_{i=0}^{j-1} \eta_i$.

In addition we need to compute the expectation, which is the sum over the expressions

\begin{equation}
\int_{k_i}^{k_{i+1}} s p_i(s) ds = k_{i+1} \eta_i - \left( \frac{1}{20}a_j(\delta_i)^5 + ... + \frac{1}{2}d_j(\delta_i)^2\right)
\end{equation}

for $i=0,...,n-1$ where $\delta_i = k_{i+1} - k_i$.


\begin{thebibliography}{0}

\end{thebibliography}

\end{document}



